% $Log: abstract.tex,v $
% Revision 1.1  93/05/14  14:56:25  starflt
% Initial revision
%
% Revision 1.1  90/05/04  10:41:01  lwvanels
% Initial revision
%
%
%% The text of your abstract and nothing else (other than comments) goes here.
%% It will be single-spaced and the rest of the text that is supposed to go on
%% the abstract page will be generated by the abstractpage environment.  This
%% file should be \input (not \include 'd) from cover.tex.

A novel combined diagnostic capable of measuring multiscale
density fluctuations that extend from
magnetohydrodynamic (MHD) scales to
the lower bound of the electron temperature gradient (ETG) mode
has been designed, installed, and operated at the \diiid\space tokamak.
The combined diagnostic was constructed by adding a heterodyne interferometer
to the pre-existing phase contrast imaging (PCI) system,
both of which measure line-integrated electron-density fluctuations.
The port-space footprint is minimized by using
a single $\SI{10.6}{\micro\meter}$ CO$_2$ laser and
a single beampath.
With temporal bandwidths in excess of $\SI{1}{\mega\hertz}$,
the PCI measures high-$k$
($\SI{1.5}{\per\centi\meter} < |k| \leq \SI{25}{\per\centi\meter}$)
fluctuations with
sensitivity $3 \times 10^{13} \; \text{m}^{-2} / \sqrt{\text{kHz}}$, while
the interferometer simultaneously measures low-$k$
($|k| < \SI{5}{\per\centi\meter}$) fluctuations with
sensitivity $3 \times 10^{14} \; \text{m}^{-2} / \sqrt{\text{kHz}}$.
The intentional mid-$k$ overlap
has been empirically verified with sound-wave calibrations and
should allow quantitative investigation of the cross-scale coupling
predicted to be significant in the reactor-relevant $T_e \approx T_i$ regime.

The combined PCI-interferometer was operated during an experiment
in which the ETG drive $a / L_{T_e}$ and
the ion temperature gradient (ITG) drive $a / L_{T_i}$
were locally modified in an attempt to elicit
a multiscale turbulent response.
Numerous turbulent branches are observed.
In particular, the interferometer measures
a low-$k$ electromagnetic mode driven unstable by collisionality,
properties consistent with the micro-tearing mode (MTM), and
the PCI measures a turbulent mode
that exhibits distinct ``spectral flattening''
when increasing $a / L_{T_e}$ relative to $a / L_{T_i}$,
hypothesized to be a tell-tale signature
of increased cross-scale coupling.
Linear-stability analysis and quasilinear-transport modeling
are performed with the trapped gyro-Landau fluid code TGLF, and
qualitative agreement with the PCI-measured spectral flattening
is obtained.

Further, via toroidal correlation with \diiid's primary interferometer,
the measurement of core-localized MHD toroidal mode numbers
has been demonstrated.
Where comparisons can be made with magnetic probes,
the interferometer-measured toroidal mode numbers
are typically in good agreement.
Unfortunately, the $\SI{4}{\centi\meter}$ major-radial offset
between the interferometer beam centers in \diiid\space
can bias the mode-number measurement,
limiting widespread use of this capability
until a robust compensation technique is developed.
