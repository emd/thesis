% $Log: abstract.tex,v $
% Revision 1.1  93/05/14  14:56:25  starflt
% Initial revision
%
% Revision 1.1  90/05/04  10:41:01  lwvanels
% Initial revision
%
%
%% The text of your abstract and nothing else (other than comments) goes here.
%% It will be single-spaced and the rest of the text that is supposed to go on
%% the abstract page will be generated by the abstractpage environment.  This
%% file should be \input (not \include 'd) from cover.tex.

A combined phase contrast imaging (PCI) and heterodyne interferometer system
has been implemented on DIII-D, extending the physics capabilities of
the pre-existing PCI and
acting as a prototypical fluctuation diagnostic for next-step devices.
The combined PCI-interferometer uses a single \SI{10.6}{\micro\meter}
laser beam, two interference schemes, and two detectors to measure
$\int \tilde{n}_e dl$ over a large spatiotemporal bandwidth
($\SI{10}{\kilo\hertz} < f < \SI{2}{\mega\hertz}$ and
$0 \leq k \leq \SI{20}{\centi\meter}^{-1}$),
allowing simultaneous measurement of ion- and electron-scale instabilities.
Further, time-correlating our interferometer's measurements with
those of DIII-D's pre-existing, toroidally separated
($\Delta \zeta = 45^{\circ}$) interferometer will allow
novel studies of low-$n$ Alfv\'{e}n eigenmodes.
The combined diagnostic's small port requirements and
minimal access restrictions make it well-suited to
the harsh neutron environments and limited port space
expected in next-step devices.
Measurements from sound wave calibrations and DIII-D operations
will be presented.
