\chapter{Design considerations for a heterodyne interferometer}
\label{ch:DesignConsiderations}


\section{Geometric effects of unmatched beams}
The external reference-beam interferometry derivations
in Section~\ref{sec:InterferometricMethods:interferometry}
assumed that the reference beam was exactly matched
in both amplitude and spatial structure
to the unscattered probe beam.
This is obviously an idealization
that, at best, can only be approached asymptotically in experiment.
This section discusses the geometric effects
of such imperfections in beam matching.

The derivation of the heterodyne intensity
(\ref{eq:InterferometricMethods:heterodyne_intensity})
can be easily generalized to account for
the geometric effects of unmatched reference and probe beams.
Namely, let the image-plane probe radiation be given by
\begin{equation}
  E_P(\vect{r}_{\image}, t)
  \approx
  E_{G,P}(\vect{r}_{\image}, t)
  e^{i \bar{\phi}}
  \left[%
    1
    +
    i \tilde{\phi}_0 \cos\nu
  \right],
\end{equation}
and let the corresponding reference beam be given by
\begin{equation}
  E_R(\vect{r}_{R}, t)
  =
  E_{G,R}(\vect{r}_{R}, t) e^{-i \Delta\omega_0 t},
\end{equation}
where $\vect{r}_{\image} = (x_{\image}, y_{\image}, z_{\image})$,
\begin{equation}
  \vect{r}_{R}
  =
  \vect{r}_{\image}
  +
  (0, 0, z_{R} - z_{\image}),
\end{equation}
and $E_{G,j}$ is a Gaussian beam
with angular frequency $\omega_0$,
waist amplitude $E_{0,j}$, and
waist 1/e $E$ radius $w_{0,j}$.
If $z_R \neq z_{\image}$,
the reference beam's waist sits at a different location
than that of the unscattered probe beam.
Under these circumstances, the heterodyne intensity becomes
\begin{equation}
  \begin{aligned}
    I_{\text{het}}(\vect{r}_{\image}, z_R, t)
    =
    2 I_{G,P}(\vect{r}_{\image})
    \bigl[%
      &\alpha_{\text{DC}}
      +
      \alpha_{\text{AC}}
      \cos(\Delta \omega_0 t + \bar{\phi}_{\text{eff}})
      \\
      &-
      \tilde{\phi}_0 \alpha_{\text{AC}}
      \sin(\Delta \omega_0 t + \bar{\phi}_{\text{eff}}) \cos\nu
    \bigr],
  \end{aligned}
  \label{eq:DesignConsiderations:heterodyne_intensity}
\end{equation}
where
\begin{equation}
  \bar{\phi}_{\text{eff}}
  =
  \bar{\phi}
  +
  \bigl[ \phi_{G,P}(\vect{r}_{\image}) - \phi_{G,R}(\vect{r}_R) \bigr]
\end{equation}
is the effective bulk phase,
\begin{equation}
  \phi_{G,j}(\vect{r})
  =
  k_0 z + \frac{k_0 \rho^2}{2 R_j(z)} - \psi_j(z)
\end{equation}
is the phase of Gaussian beam $j \in \{P, R\}$
(i.e.\ $E_{G,j}(\vect{r}) = |E_{G,j}(\vect{r})| e^{i \phi_{G,j}(\vect{r})}$),
\begin{align}
  \alpha_{\text{DC}}
  &=
  \frac{1}{2}\left[%
    1
    +
    \frac{I_{G,R}(\vect{r}_R)}{I_{G,P}(\vect{r}_{\image})}
  \right],
  \\
  \alpha_{\text{AC}}
  &=
  \sqrt{\frac{I_{G,R}(\vect{r}_R)}{I_{G,P}(\vect{r}_{\image})}},
\end{align}
are geometric factors that describe the amplitudes
of the DC and AC components of the heterodyne signal, and
\begin{equation}
  I_{G,j}(\vect{r})
  =
  \frac{c \varepsilon_0 |E_{G,j}(\vect{r})|^2}{2}
\end{equation}
is the intensity profile (averaged over an optical cycle)
of Gaussian beam $j \in \{P, R\}$.
Note that (\ref{eq:DesignConsiderations:heterodyne_intensity}) readily reduces to
(\ref{eq:InterferometricMethods:heterodyne_intensity})
if $E_{G,R}(\vect{r}_R) = E_{G,P}(\vect{r}_{\image})$.

It is worth discussing the implications of heterodyne intensity
(\ref{eq:DesignConsiderations:heterodyne_intensity}).
First, the AC component of the intensity
carries the desired phase information, and
maximizing the ratio of the AC signal to the DC signal requires that
$I_{G,R}(\vect{r}_R) = I_{G,P}(\vect{r}_{\image})$.
Second, note that the effective bulk phase $\bar{\phi}_{\text{eff}}$
is dependent on the geometry of the reference beam and
the unscattered probe beam.
Specifically, in the context of measuring
the plasma-induced bulk phase $\bar{\phi}$,
note that
\begin{equation}
  \bar{\phi}_{\text{eff}}(\rho_{\image}=0)
  =
  \bar{\phi}
  +
  k_0 (z_{\image} - z_R)
  -
  \left[ \psi_P(z_{\image}) - \psi_R(z_R) \right].
\end{equation}
If $z_{\image}$ and $z_R$ are fixed,
then the beam-geometry contributions to
$\bar{\phi}_{\text{eff}}(\rho_{\image} = 0)$
constitute an unimportant DC offset that can be removed
via baseline subtraction;
however, experiments are typically plagued by vibrations, and
even small changes to $z_{\image}$ and $z_R$
can make significant time-dependent contributions to
$\bar{\phi}_{\text{eff}}(\rho_{\image} = 0)$ at CO$_2$ probe wavelengths.
As such, deconvolving the plasma-induced and vibration-induced contributions
to $\bar{\phi}_{\text{eff}}(\rho_{\image} = 0)$
requires interferometric measurements
at two distinct wavelengths (i.e.\ two-color interferometry)
\cite{carlstrom_rsi88}.
However, such vibrations occur on slow time-scales
(e.g.\ $f_{\text{vib}} \lesssim \SI{5}{\kilo \hertz}$),
and phase measurements at a \emph{single} wavelength are sufficient
to quantify plasma-induced phase fluctuations
at frequencies above $f_{\text{vib}}$
\cite{vanzeeland_ppcf05}.
Note that the beam geometry also imparts
a spatially dependent, curvature-induced phase shift
\begin{align}
  \delta\phi_{\kappa}(\rho_{\image})
  &=
  \bar{\phi}_{\text{eff}}(\rho_{\image})
  -
  \bar{\phi}_{\text{eff}}(\rho_{\image} = 0)
  \notag \\
  &=
  \frac{k_0 \rho_{\image}^2}{2}
  \left[\frac{1}{R_P(z_{\image})} - \frac{1}{R_R(z_R)} \right],
\end{align}
which can result in signal loss and distortion of the measured wavenumber.
To see this, assume that the radiation is interfered on a detector array,
as shown in Fig.~\ref{fig:InterferometricMethods:detector_array}.
As a detector element produces a signal
proportional to the average intensity across its face,
there will be substantial signal loss
if there are large curvature-induced phase shifts
across the element's face
(i.e.\ $\delta\phi_{\kappa}(s_x / 2) \gtrsim \pi$ or
$\delta\phi_{\kappa}(s_y / 2) \gtrsim \pi$).
Further, if there are large curvature-induced phase shifts
across the length of the detector array,
the spatial structure of the intensity
will \emph{not} correspond to the spatial structure
of the plasma fluctuation.
The latter is the more conservative constraint
on the curvature-induced phase shift.
Assuming that the detector array shown in
Fig.~\ref{fig:InterferometricMethods:detector_array}
consists of $N_{\text{el}}$ detector elements and
that the inter-element spacing is negligible ($\delta_x \ll s_x$),
the criterion for negligible curvature-induced phase shifts
$\delta\phi_{\kappa, \text{max}}
=
\delta\phi_{\kappa}(\rho_{\image, \text{max}})
\ll
1$
becomes
\begin{equation}
  \frac{k_0}{8}
  \left[ (N_{\text{el}} s_x)^2 + s_y^2 \right]
  \left| \frac{1}{R_P(z_{\image})} - \frac{1}{R_R(z_R)}\right|
  \ll
  \pi.
\end{equation}


\section{Phase noise: sources and effects}


\subsection{Umatched optical path lengths}
The external reference-beam interferometry derivations
in Section~\ref{sec:InterferometricMethods:interferometry}
assumed that the laser's angular frequency was fixed
at its nominal value $\omega_0$.
However, the angular frequency of any \emph{real} laser
will exhibit short-term jitter and long-term drift,
much like any other real-world oscillator
\cite[Sec.~1.7]{siegman_lasers}.
The electric field of a Gaussian beam
with short-term frequency jitter
is well-described by
\begin{equation}
  E_G(\vect{r}, t)
  =
  E_G(\vect{r})
  e^{-i [\omega_0 t + \phi_{\omega_0}(t)]}
\end{equation}
where $\phi_{\omega_0}(t)$ is a zero-mean, stationary, random process
whose temporal variation causes
the oscillator's instantaneous angular frequency
to wander about its nominal value $\omega_0$.

Now, if the interferometer's probe beam and reference beam
traverse different optical path lengths,
the laser's short-term jitter will inject
phase noise into the measured signal.
To see this, assume that the optical path length of the probe beam
exceeds that of the reference arm by $L$.
Then, if the reference beam impinging on the detector at time $t$ is
\begin{equation}
  E_R(\vect{r}_{\image}, t)
  =
  E_G(\vect{r}_{\image})
  e^{-i [
    (\omega_0 + \Delta \omega_0) t
    +
    \phi_{\omega_0}(t)
  ]},
\end{equation}
the corresponding imaged probe radiation is
\begin{equation}
  E_P(\vect{r}_{\image}, t)
  =
  E_G(\vect{r}_{\image})
  e^{-i [\omega_0 (t - \tau) + \phi_{\omega_0}(t - \tau)]}
  e^{i \bar{\phi}}
  \left[%
    1
    +
    i \tilde{\phi}_0 \cos\nu
  \right],
\end{equation}
where $\tau = L / c$ is the time delay
associated with the optical path-length difference $L$.
Define the jitter-induced phase difference
\begin{equation}
  \delta \phi_{\omega_0}(t, \tau)
  =
  \phi_{\omega_0}(t + \tau)
  -
  \phi_{\omega_0}(t).
\end{equation}
Typically, $\delta \phi_{\omega_0}(t, \tau) \ll 1$.
Then, appropriately generalizing the derivations between
(\ref{eq:InterferometricMethods:heterodyne_intensity}) and
(\ref{eq:InterferometricMethods:heterodyne_total_fluctuating_power_per_element}),
one readily finds that the total fluctuating power
in the heterodyne interferometer's demodulated signals is
\begin{equation}
  \tilde{P}_{j, IQ}(t)
  =
  I_G(\vect{r}_{\image,j}) A
  \left[%
    T_{\text{fsv}}(k_{\image})
    \tilde{\phi}_0
    \cos\nu_j
    +
    \delta \phi_{\omega_0}(t - \tau, \tau)
  \right];
\end{equation}
that is, the fluctuating signal is contaminated
by the laser's jitter.

\begin{itemize}
  \item \textcolor{red}{Compute autospectral density and compare to ideal}
\end{itemize}


\subsection{Finite modulator coupling time}
\begin{itemize}
  \item \textcolor{red}{similar to laser phase-noise discussion}
  \item \textcolor{red}{Compute autospectral density and compare to ideal}
\end{itemize}


\section{Amplitude noise: sources and effects}
\begin{itemize}
  \item Electronics following demodulation equally affect signal and noise, so
    emphasis here is on noise immediately following demodulation
  \item Laser intensity fluctuations on heterodyne time scale negligible (?)
\end{itemize}

\subsection{Detector noise}

\subsection{Shot noise}


\section{Demodulator imperfections}


\section{Bit noise}


\bibliographystyle{plainurl}
\bibliography{references}
