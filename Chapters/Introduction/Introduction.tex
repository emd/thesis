\chapter{Introduction}


\section{Fusion energy}
Fusion is the energy of the sun and stars, and,
for the better part of the last century,
man has been striving to replicate this process
in a controlled manner here on earth
for a source of safe, clean, virtually inexhaustible source of energy.


\section{Tokamak fusion}
\subsection{Turbulent transport}
\subsection{Energetic-particle confinement}


\section{Interferometry}


\section{Thesis outline}
The remainder of this thesis is organized as follows:

\begin{itemize}
  \item Chapter 2 discusses the theory of optical interferometric methods
    in the context of measuring tokamak-plasma-density fluctuations.
    The laser-plasma interaction is quantified via
    Fraunhofer scalar-diffraction theory, and
    the resulting diffracted field is imaged on a square-law detector.
    Interfering the imaged field with a known reference field
    produces measurable intensity fluctuations;
    the specification of this reference field
    defines the interferometric method.
    Details of two particular interferometric methods ---
    external-reference-beam interferometry and
    phase contrast imaging (PCI) ---
    are discussed, with an emphasis on
    their sensitivity to fluctuations and their spatiotemporal bandwidths.
    Significantly, while PCI can measure fluctuations more sensitively
    than an external-reference-beam interferometer,
    PCI suffers from a low-$k$ cutoff;
    an external-reference-beam interferometer
    does \emph{not} suffer from such a low-$k$ cutoff.
  \item Chapter 3 considers the design of an
    external-reference-beam, heterodyne interferometer
    (hereafter referred to as a heterodyne interferometer).
    A criterion for satisfactory wavefront matching
    between the probe beam and the reference beam is developed, and
    finite-sampling-volume effects are shown
    to constrain the heterodyne interferometer's spatial bandwidth.
    The effects of phase noise, amplitude noise, and digitizer bit noise
    are each discussed in the context of
    the heterodyne interferometer's signal-to-noise ratio, and
    the systematic errors resulting from
    imperfect demodulation of the heterodyne interference signal
    are quantified.
  \item Chapter 4
  \item Chapter 5
  \item Chapter 6
  \item Chapter 7
\end{itemize}


\section{Units}
