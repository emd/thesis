\chapter{Introduction}
\label{ch:Introduction}


\section{Fusion energy}
Fusion is the energy of the Sun and stars, and,
for the better part of the last century,
thousands of scientists and engineers have striven
to replicate this process in a controlled manner here on Earth
for a source of safe, clean, and virtually inexhaustible energy.
The core principle is simple:
combine (i.e. ``fuse'') the nuclei of two light elements
to make the nucleus of a heavier element,
with the total mass of the resulting products
being slightly less than that of the reactants;
this mass difference $m$
is converted into an enormous amount of energy $E$
via Einstein's famed $E = m c^2$, where
$c$ is the speed of light in vacuum~\cite[Ch.~14]{krane}.

Practically speaking, however, fusion is extremely difficult.
Even for deuterium (D) and tritium (T), the most reactive of nuclei,
the probability of fusing
(quantified by the reaction cross section)
is orders-of-magnitude smaller than
the probability of the positively charged nuclei
repelling and scattering off of each other
(quantified by the Coulomb cross section)
\cite[Sec.~9.3.4]{freidberg_fusion_energy}.
This relatively small fusion cross section explains why
beam-target or beam-beam fusion are incapable of producing net energy.
Instead of using a highly directed beam that is far from thermal equilibrium,
an alternative approach is to employ the thermal energy
of a material at or near thermal equilibrium
to overcome the Coulomb barrier ---
because of its reliance on the thermal energy,
such an approach is referred to as thermonuclear fusion~\cite[Ch.~14]{krane}.

The temperatures required to initiate thermonuclear fusion
dictate that the fuel exists as a hot, ionized gas
referred to as plasma.
Initially, the plasma must be externally heated
to reach fusion-relevant conditions, but
it is envisioned that at some threshold, much like the Sun,
the fusion reactions will become self-sustaining;
this threshold is referred to as \emph{ignition}.
Now, in addition to the thermal energy
required to overcome the Coulomb barrier,
the plasma must also be sufficiently dense and
sufficiently well-confined
to reach ignition.
This threshold for ignition in a D-T plasma is often quantified
by the fusion triple product
\begin{equation}
  n_{i} T_{i} \tau_E
  \geq
  5 \times 10^{21} \, \text{m}^{-3} \cdot \text{keV} \cdot \text{s}
  =
  8 \, \text{atm} \cdot \text{s},
  \label{eq:Introduction:fusion_triple_product}
\end{equation}
where
$n_{i}$ is the peak ion density,
$T_{i}$ is the peak ion temperature, and
$\tau_E$ is the energy confinement time~\cite[Sec.~1.1]{wesson}.
This condition could be reached, for example, at
$n_i = \SI{e20}{\per\meter\cubed}$,
$T_i = \SI{10}{\kilo\eV}$, and
$\tau_E = \SI{3}{\second}$~\cite[Sec.~1.3]{wesson}.

It should be emphasized that $T_i = \SI{10}{\kilo\eV}$
corresponds to approximately $\SI{e6}{\kelvin}$.
No conventional container can confine such a plasma.
While numerous confinement schemes exist,
the only strategy considered in this thesis
is that of tokamak magnetic confinement,
briefly summarized in
Section~\ref{sec:Introduction:FusionEnergy:tokamaks}
While the tokamak's magnetic fields
prevent the hot, core plasma from contacting any material walls,
the outermost portions of the plasma (i.e. ``the edge'')
do contact the reactor walls.
To prevent damage to the reactor walls,
the plasma temperature and density
must decrease to tolerable levels at the edge;
the resulting gradients in density and temperature
provide free energy for numerous coherent and broadband instabilities,
some of which are measured and characterized in this thesis.


\subsection{The tokamak approach to fusion}
\label{sec:Introduction:FusionEnergy:tokamaks}
The tokamak~\cite{wesson} is currently
the leading configuration for a magnetic-confinement fusion reactor.
The tokamak is an axisymmetric toroidal device
characterized by a strong toroidal magnetic field $B_{\zeta}$ and
a toroidal plasma current $I_{\zeta}$.
The toroidal plasma current
produces a poloidal magnetic field $B_{\theta}$,
resulting in a total magnetic field
$\vect{B} = B_{\zeta} \hat{\vect{\zeta}} + B_{\theta} \hat{\vect{\theta}}$
that wraps helically around the torus
in a barber pole-like fashion.
The magnetic force of this helical field balances the plasma pressure,
allowing the establishment of an equilibrium;
the large toroidal field $B_{\zeta}$ provides stability.
Wesson provides a concise historical overview of tokamak research
\cite[Sec.~1.10]{wesson}.
Since the initial success of the USSR's T-3 tokamak in the 1960s,
hundreds of tokamaks of numerous shapes, sizes, and field strengths
have been built all over the world
\cite[Ch.~11,12]{wesson}~\cite{tokamaks_of_the_world}.

The work reported in this thesis was conducted
at the \diiid\space tokamak in San Diego, CA~\cite[Sec.~12.5]{wesson}.
\diiid\space is a mid-size
($R = \SI{1.67}{\meter}$, $a = \SI{0.67}{\meter}$)
diverted tokamak with
maximum toroidal field $B_{\zeta} \leq \SI{2.2}{\tesla}$ and
a maximum achieved plasma current of $I_{\zeta} \leq \SI{3}{\mega\amp}$.
Up to $\SI{20}{\mega\watt}$ of deuterium neutral beam injection (NBI) and
$\SI{6}{\mega\watt}$ of electron cyclotron resonance heating (ECRH)
are available for auxiliary plasma heating.
The ECRH deposition location can be dynamically changed within a single shot,
and the NBI system can be operated in $\beta$ feedback
in an attempt to maintain constant plasma $\beta$
as other plasma parameters change.
\diiid\space is perhaps the best-diagnosed tokamak in the world,
with an extensive suite of equilibrium and fluctuation diagnostics.


\subsection{Fluctuation-induced transport in tokamak plasmas}
As hinted at above, the gradients in a fusion plasma can drive instability.
These instabilities may be broadband or coherent in nature, and
they may be beneficial, benign, or detrimental to confinement.
The brief discussion below is \emph{not}
a thorough or exhaustive survey of these instabilities but
is instead intended to provide a fusion-energy contex
to the diagnostic development pursued in this thesis.

The radial transport of particles, heat, and momentum in a tokamak plasma is
often larger than that predicted by collisional (i.e.\ neoclassical) theory.
There is strong evidence that this ``anomalous'' transport
results from electrostatic drift-wave turbulence
driven by the free energy in the plasma gradients~\cite{horton_drift_waves}.
Due to their large eddy size,
ion-scale ($k_{\theta} \rho_i \lesssim 1$) turbulence is often considered
to be the most detrimental to confinement~\cite{horton_drift_waves}, but
electron-scale ($k_{\theta} \rho_e \lesssim 1$) turbulence
may be capable of forming radially elongated ``streamers''
\cite{dorland_prl00}
capable of driving experimentally relevant electron heat transport
\cite{jenko_prl02}.
Here, $k_{\theta}$ is a typical poloidal wavenumber of the turbulent mode,
$\rho_j = v_{tj} / \Omega_j$ is the gyroradius of species $j$,
$v_{tj} = \sqrt{2 T_j / m_j}$ is the thermal speed of species $j$, and
$\Omega_j = e B / m_j$ is the angular cyclotron frequency of species $j$.
For the reactor-relevant scenario $T_e \sim T_i$,
the characteristic length scale of ion-scale turbulence is a factor
$\sqrt{m_D / m_e} \approx 60$ larger than
the characteristic length scale of electron-scale turbulence,
where the $m_D$ is the deuteron mass.
Until the very recent work of Howard \emph{et al.}
\cite{howard_pp14,howard_nf16},
self-consistently and simultaneously simulating
both ion- and electron-scale turbulence with realistic mass ratios
was computationally intractable.
In a computational \emph{tour de force},
Howard \emph{et al.}'s multiscale simulations indicate
that cross-scale coupling can drive
experimentally relevant levels of electron heat flux~\cite{howard_pp14} and
that this cross-scale coupling becomes stronger
when the ion-scale turbulence is marginally (rather than strongly) unstable
\cite{howard_nf16}.

In addition to broadband turbulence,
coherent fluctuations can also drive transport in a tokamak plasma.
Of note are Alfv\'{e}n waves that are driven unstable
by resonant interactions with superthermal energetic particles
\cite{wesson,heidbrink_pp08}.
Such energetic particles sit in the tail
of the plasma's distribution function, and
they are generated, for example, from
the ionization of injected neutral-beam particles,
the acceleration of ions by intense radio-frequency fields, or
the fusion of two fuel ions.
Nonlinearly, the Alfv\'{e}n instability
results in the loss of energetic particles from the plasma,
decreasing the fusion rate;
further, these lost particles may damage components inside the reactor.
(As an interesting aside, Alfv\'{e}n spectroscopy is an effective diagnostic
for measuring the minimum value of the plasma's safety factor,
$q_{\text{min}}$~\cite{wesson,edlund_prl09,breizman_pp05}).
Of much practical interest to stable tokamak operation
is the neoclassical tearing mode~\cite[Sec.~7.3]{wesson},
which can ``lock'' to the vessel wall~\cite[Sec.~7.10]{wesson}
and result in ``disruption'',
an abrupt, violent termination of the plasma discharge
\cite[Sec.~7.7-7.9]{wesson}.


\section{Optical interferometry}
Following its invention by Michelson in the 1880s~\cite{nobel_prize_michelson},
optical interferometry has had a long and illustrious history
in fundamental and applied sciences, ranging from
the famed Michelson-Morley experiment
disproving the existence of a luminiferous ether~\cite{michelson_ajs1887},
to Zernike's phase-contrast method for imaging phase objects
of wide importance in biology~\cite{nobel_prize_zernike},
to the recent observation of of gravitational waves
by the LIGO collaboration~\cite{ligo_prl16}.
The brief discussion of interferometry below is intended
to provide an instrumentation context
to the diagnostic development pursued in this thesis and
to equip the reader with a foundation
for the more detailed discussion in
Chapter~\ref{ch:InterferometricMethods}.

Interferometry exploits the interference of light
to measure optical path length, defined as the product of
the geometric path length $l$ and
the index of refraction $N$ of the intervening medium.
The phase $\delta \phi$ acquired by light propagating through
optical path length $\delta(Nl)$ is
$\delta\phi = k \delta(NL)$, where
$k$ is the in-medium wavenumber of the light;
integrating over the geometric path
yields the total acquired phase, $\phi$.
Thus, an incident electric field $E_0$
will be phase shifted as $E_0 e^{i \phi}$
after transiting the above optical path.
Note that this phase shift does \emph{not} alter
the field intensity $I \propto |E_0 e^{i \phi}|^2 = |E_0|^2$;
as most detectors are square-law, intensity detectors,
such naive detection of the phase-shifted radiation
provides no information about the phase $\phi$
or the underlying optical path.
However, by interfering the phase-shifted field with
a reference field $E_R = E_0 e^{i \phi_R}$ of
known amplitude $E_0$ and known phase $\phi_R$,
the total field becomes
\begin{equation}
  E_{\text{tot}}
  =
  E_R + E_0 e^{i \phi}
  =
  E_0 \left( e^{i \phi_R} + e^{i \phi} \right),
  \label{eq:Introduction:OpticalInterferometry:generic_interferometer_electric_field}
\end{equation}
with corresponding intensity
\begin{equation}
  I
  \propto
  |E_{\text{tot}}|^2
  =
  2 E_0^2 \left[
    1 + \cos\left( \phi - \phi_R \right)
  \right].
  \label{eq:Introduction:OpticalInterferometry:generic_interferometer_intensity}
\end{equation}
Now, the measurable intensity $I$
is a function of the phase $\phi$.
The specification of the reference field $E_R$
determines the interferometric method.
Below, two interferometric methods ---
heterodyne interferometry and phase contrast imaging (PCI) ---
are briefly discussed.


\subsection{Heterodyne interferometry}
Heterodyne interferometry~\cite{hutchinson_diagnostics}
uses an external reference beam
whose angular frequency has been shifted by $\Delta \omega_0$
(i.e.\ the reference field is $E_R = E_0 e^{-i \Delta \omega_0 t}$)
to produce measurable intensity variations
\begin{equation}
  I_{\text{het}}
  \propto
  2E_0^2 \left[
    1 + \cos \left(\Delta \omega_0 t + \phi \right)
  \right].
  \label{eq:Introduction:OpticalInterferometry:Heterodyne:intensity}
\end{equation}
Note that the desired baseband phase information $\phi$ is shifted
to an intermediate frequency $\Delta \omega_0$;
quadrature heterodyne detection can then be used
to extract an absolute measurement of~$\phi$~\cite{carlstrom_rsi88}.
The technical complications of converting to and from
the intermediate frequency
are often justified by the resulting ability
to overcome the shortcomings of homodyne interferometry
\cite{hutchinson_diagnostics, nazikian_rsi87}.
Heterodyne interferometry is an established technique
for measuring both the bulk and the fluctuating components of plasma density
in contemporary tokamaks
\cite{carlstrom_rsi88, vanzeeland_ppcf05, mlynek_fst12, kasten_rsi12}
and is expected to provide similar capabilities
in ITER~\cite{vanzeeland_TIP_rsi13} and other next-generation devices.


\subsection{Phase contrast imaging (PCI)}
\label{sec:Introduction:OpticalInterferometry:pci}
To motivate the development of phase contrast imaging (PCI),
consider a phase $\phi$ that consists of a
uniform, bulk contribution $\bar{\phi}$ and
a small, spatially fluctuating contribution $\tilde{\phi} \ll 1$ such that
$\phi = \bar{\phi} + \tilde{\phi}$.
To lowest order in $\tilde{\phi}$,
the heterodyne intensity
(\ref{eq:Introduction:OpticalInterferometry:Heterodyne:intensity})
becomes
\begin{equation}
  I_{\text{het}}
  \propto
  2 E_0^2 \left\{
    1
    +
    \left[
      \cos \left(\Delta \omega_0 t + \bar{\phi} \right)
      -
      \tilde{\phi} \sin \left( \Delta \omega_0 t + \bar{\phi} \right)
    \right]
  \right\}.
\end{equation}
As $\tilde{\phi} \ll 1$,
the component of the intensity
corresponding to the fluctuation $\tilde{\phi}$
is only a \emph{small fraction} of the total intensity.
Because every physical detector
has a maximum tolerable intensity $I_{\text{max}}$,
this implies that the fluctuation-induced component of the signal
will only occupy a small fraction
of the detector's dynamic range,
establishing a fundamental limit
on the fluctuation sensitivity of a heterodyne interferometer.

If measurement of the fluctuation $\tilde{\phi}$ is of primary importance,
the interference scheme can be reconfigured
to provide better sensitivity to fluctuations.
Examine the phase-shifted field
\begin{equation}
  E_0 e^{i \phi}
  =
  E_0 e^{i \left( \bar{\phi} + \tilde{\phi} \right)}
  \approx
  E_0 e^{i \bar{\phi}} \left( 1 + i \tilde{\phi} \right),
\end{equation}
where only the lowest-order term in $\tilde{\phi}$ has been retained.
In the rightmost expression,
the unity term corresponds to the bulk-phase contribution, while
the $i \tilde{\phi}$ term corresponds to the fluctuating-phase contribution.
Now, imagine prescribing a reference field
\begin{equation}
  E_R = -E_0 e^{i \bar{\phi}} (1 - i \eta)
\end{equation}
for $0 < \eta \leq 1$ such that the total field becomes
\begin{equation}
  E_{\text{tot}}
  =
  E_R + E_0 e^{i \phi}
  \approx
  i E_0 e^{i \bar{\phi}} \left( \eta + \tilde{\phi} \right),
\end{equation}
with corresponding intensity
\begin{equation}
  I_{\text{PCI}}
  \propto
  |E_{\text{tot}}|^2
  \approx
  E_0^2 \left( \eta + 2 \sqrt{\eta} \tilde{\phi} \right),
  \label{eq:Introduction:OpticalInterferometry:PCI:intensity}
\end{equation}
where only the lowest order terms in $\tilde{\phi}$
have been retained in the expressions
for both $E_{\text{tot}}$ and $I_{\text{PCI}}$.
For a given fluctuation $\tilde{\phi}$ and a given detector,
such an interference scheme has an amplitude sensitivity to fluctuations
that is $8 / \sqrt{\eta}$ \emph{better}
than that of a heterodyne interferometer.
Note, however, that information about the bulk phase $\bar{\phi}$ is lost.

While prescription of the above reference field may seem rather academic,
this is precisely the means by which PCI operates,
as may be guessed from the suggestive notation $I_{\text{PCI}}$.
PCI ``prescribes'' such a reference field
by spatially filtering the radiation pattern $E_0 e^{i \phi}$
in the focal plane of a focusing optic.
Typically, the spatial filtering is performed with a ``phase plate'',
an optical element that selectively applies
(due to its geometry and its location at the focal plane)
an appropriate phase delay and attenuation
to the component of the field corresponding to the bulk phase $\bar{\phi}$.
Thus, in contrast to the heterodyne interferometer,
the PCI uses an \emph{internal} reference beam.
Because diffraction limits the focal-plane spot size,
this spatial filtering necessarily involves a low-$k$ cutoff,
which in the ideal, diffraction-limited case corresponds to
$k_{\text{min}}^{\text{PCI}} = 2 / w$,
where $w$ is the 1/e electric field radius of the probe beam
\cite{dorris_rsi09}.
The principles of PCI will be discussed in more detail in
Section~\ref{sec:InterferometricMethods:pci}.


PCI is an established technique for measuring plasma density fluctuations
in contemporary tokamaks.
\begin{itemize}
  \item references to PCI systems (Dorris \diiid, C-Mod, TCV)
  \item List important contributions
    (Rost/Marinoni/Ennever: turbulence, Naoto: ICRF, Edlund: AEs)
  \item Comment about spatial localization via additional spatial filtering
    (Dorris RSI, Lin RSI (?)), but not pursued further in this thesis
\end{itemize}


\subsection{Heterodyne interferometry vs.\ PCI}
Colloquially, heterodyne interferometry
is considered a ``low-$k$'' technique, and
PCI is considered a ``high-$k$'' technique.
It is worth pausing here to note that,
for the same probe beam and the same fluctuation $\tilde{\phi}$,
the laser-plasma interaction is \emph{identical} for both techniques.
Further, the high-$k$ optical capabilities of both systems
are governed by the size of the collection optics
and finite sampling-volume effects~\cite{bravenec_rsi95}, as discussed in
Section~\ref{sec:DesignConsiderations:geometric:finite_sampling_volume}.
Thus, there is nothing that intrinsically limits
heterodyne interferometry to low-$k$ measurements ---
a heterodyne interferometer's high-$k$ limit
can be just as high, if not higher,
than that of a given PCI system.
However, as outlined in
Section~\ref{sec:Introduction:OpticalInterferometry:pci},
PCI is \emph{more sensitive} to fluctuations
than a comparable heterodyne interferometer, and,
assuming a Kolmogorov-like fluctuation spectrum
$S(k) \propto k^{-p}$ for some positive $p$,
PCI's superior sensitivity may allow it
to detect high-$k$ fluctuations
that are too weak to be seen by a heterodyne interferometer.
At the low-$k$ side of the spectrum,
PCI is diffraction-limited to $k \geq 2 / w$,
where $w$ is the 1/e electric field radius of the probe beam
\cite{dorris_rsi09},
whereas the heterodyne interferometer's external reference beam
allows detection even at $k = 0$.


\section{Motivation for a combined PCI-interferometer}
Development of a combined PCI-interferometer for
\begin{itemize}
  \item measurement of multiscale turbulence,
    \begin{itemize}
      \item ITG experiment: Tynan; ETG experiment: ???
      \item Multiscale predictions need validation
    \end{itemize}
  \item measurement of core-localized MHD, and
  \item diagnostic development for next-generation devices
    with limited port space
\end{itemize}


\section{Thesis outline}
The remainder of this thesis is organized as follows:

\begin{itemize}
  \item Chapter~\ref{ch:InterferometricMethods} discusses
    the theory of optical interferometric methods
    in the context of measuring tokamak-plasma-density fluctuations.
    The laser-plasma interaction is quantified via
    Fraunhofer scalar-diffraction theory, and
    the resulting diffracted field is imaged on a square-law detector.
    Interfering the imaged field with a known reference field
    produces measurable intensity fluctuations;
    the specification of this reference field
    defines the interferometric method.
    Details of two particular interferometric methods ---
    external-reference-beam interferometry and
    phase contrast imaging (PCI) ---
    are discussed, with an emphasis on
    their sensitivity to fluctuations and their spatiotemporal bandwidths.
    Significantly, while PCI can measure fluctuations more sensitively
    than an external-reference-beam interferometer,
    PCI suffers from a low-$k$ cutoff;
    an external-reference-beam interferometer
    does \emph{not} suffer from such a low-$k$ cutoff.
  \item Chapter~\ref{ch:DesignConsiderations} considers the design of an
    external-reference-beam, heterodyne interferometer
    (hereafter referred to as a heterodyne interferometer).
    A criterion for satisfactory wavefront matching
    between the probe beam and the reference beam is developed, and
    finite-sampling-volume effects are shown
    to constrain the heterodyne interferometer's spatial bandwidth.
    The effects of phase noise, amplitude noise, and digitizer bit noise
    are each discussed in the context of
    the heterodyne interferometer's signal-to-noise ratio, and
    the systematic errors resulting from
    imperfect demodulation of the heterodyne interference signal
    are quantified.
  \item Chapter~\ref{ch:Implementation} details
    the addition of a heterodyne interferometer
    to the pre-existing PCI system on the \diiid\space tokamak;
    both systems operate simultaneously,
    sharing a single $\SI{10.6}{\micro\meter}$ probe beam through the plasma.
    Optical-diagnostic access on \diiid\space and the capabilities
    of the pre-existing PCI system are briefly reviewed.
    Referencing the design considerations
    in Chapter~\ref{ch:DesignConsiderations} and
    adopting the philosophy
    that the pre-existing PCI system should be minimally perturbed,
    the optical layout for the heterodyne interferometer is developed;
    the magnification of the interferometer's imaging system
    is selected such that the spatial bandwidths
    of the PCI and interferometer have a mid-$k$ overlap.
    The design, procurement, and installation
    of the new optical and electrical components
    required to make a heterodyne interferometric measurement
    are summarized.
    Of note is the interferometer's radio-frequency local oscillator:
    the phase noise of a crystal oscillator (XO)
    was empirically found to be \emph{too large}
    to make meaningful fluctuation measurements in most tokamak plasmas, but
    the substantially lower phase noise of an
    oven-controlled crystal oscillator (OCXO)
    allows measurements of a whole zoo
    of coherent and broadband fluctuations.
    The multiscale capabilities of the combined PCI-interferometer
    are empirically verified via sound-wave calibrations.
    Finally, some remarks are made regarding
    the use of such a combined PCI-interferometer
    on next-step devices, such as ITER.
  \item Chapter~\ref{ch:ToroidalCorrelation} discusses the correlation
    of the newly installed interferometer with
    \diiid's toroidally separated, pre-existing V2 interferometer.
    Capable of probing the core plasma,
    the interferometers are shown to be sensitive
    to core-localized fluctuations
    that are \emph{invisible} to external magnetic probes.
    The chapter begins with a brief review
    of the two-point correlation technique and
    shows how toroidal mode numbers can be extracted
    from a pair of toroidally separated measurements.
    Meaningful correlation requires that
    the two measurements share the same timebase.
    The digitizers of both interferometers were modified
    to phase lock their clocks, and
    a residual ``trigger offset'' was measured
    and is compensated in software.
    Where comparisons can be made with magnetic probes,
    the interferometer-measured toroidal mode numbers
    are in good agreement.
    Implications of the \SI{4}{\centi\meter} radial offset
    between the beam centers of each interferometer are discussed.
  \item Chapter~\ref{ch:TurbulenceMeasurements} demonstrates
    the ability of the combined PCI-interferometer
    to measure multiscale, broadband fluctuations.
    During a recent \diiid\space experiment,
    the location of electron-cyclotron heating (ECH)
    was moved from $\rho = 0.5 \rightarrow 0.8$,
    altering $T_e / T_i$
    in an attempt to change the coupling between
    the electron-scale and ion-scale turbulence.
    At low frequencies ($f < \SI{200}{\kilo\hertz}$),
    the low-$k$ and high-$k$ responses to the change in ECH location
    are similar;
    however, at higher frequencies, they are \emph{distinct}.
    Measurements are compared to linear predictions
    from the gyro-Landau-fluid code TGLF.
  \item Chapter~\ref{ch:Conclusions} concludes the thesis
    with a summary and discussion of the primary results
    and suggests avenues of future work.
\end{itemize}


\section{Units}
Unless explicitly stated, all formulas in this thesis are written in SI units.
The one notable exception, of course,
is the suppression of the Boltzmann constant
in favor of expressing temperatures in units of energy.


\bibliographystyle{plainurl}
\bibliography{references}
