\chapter{Introduction}
\label{ch:Introduction}


\section{Fusion energy}
Fusion is the energy of the Sun and stars, and,
for the better part of the last century,
thousands of scientists and engineers have striven
to replicate this process in a controlled manner here on Earth
for a source of safe, clean, and virtually inexhaustible energy.
The core principle is simple:
combine (i.e. ``fuse'') the nuclei of two light elements
to make the nucleus of a heavier element,
with the total mass of the resulting products
being slightly less than that of the reactants;
this mass difference $m$
is converted into an enormous amount of energy $E$
via Einstein's famed $E = m c^2$, where
$c$ is the speed of light in vacuum~\cite[Ch.~14]{krane}.

Practically speaking, however, fusion is extremely difficult.
Even for the most reactive of nuclei,
the probability of two positively charged nuclei repelling each other
(quantified by the Coulomb cross section) is
\textcolor{red}{orders-of-magnitude} larger
than the probability of the nuclei fusing
(quantified by the reaction cross section).
This relatively small fusion cross section explains why
beam-target or beam-beam fusion are incapable of producing net energy.
Instead of using a highly directed beam far from thermal equilibrium,
an alternative approach is to employ the thermal energy
of a material at or near thermal equilibrium
to overcome the Coulomb barrier ---
because of its reliance on the thermal energy,
such an approach is referred to as thermonuclear fusion~\cite[Ch.~14]{krane}.

The temperatures required to initiate thermonuclear fusion
dictate that the fuel exists as a hot, ionized gas
referred to as plasma.
Initially, the plasma must be externally heated
to reach fusion-relevant conditions, but
it is envisioned that at some threshold, much like the Sun,
the fusion reactions will become self-sustaining;
this threshold is referred to as \emph{ignition}.
Now, in addition to the thermal energy
required to overcome the Coulomb barrier
the plasma must also be sufficiently dense and
sufficiently well-confined
to reach ignition.
This threshold for ignition is often quantified
by the \textcolor{red}{fusion triple product}
\begin{equation}
  n_{i} T_{i} \tau_E
  \geq
  5 \times 10^{21} \text{m}^{-3} \cdot \text{keV} \cdot \text{s},
  \label{eq:Introduction:fusion_triple_product}
\end{equation}
where
$n_{i}$ is the peak ion density,
$T_{i}$ is the peak ion temperature, and
$\tau_E$ is the energy confinement time~\cite[Sec.~1.1]{wesson}.
This condition could be reached, for example, at
$n_i = \SI{1e20}{\per\meter\cubed}$,
$T_i = \SI{10}{\kilo\eV}$, and
$\tau_E = \SI{3}{\second}$~\cite[Sec.~1.3]{wesson}.

It should be emphasized that $T_i = \SI{10}{\kilo\eV}$
corresponds to approximately $\SI{e6}{\kelvin}$.
No conventional container can confine such a plasma.
While numerous confinement schemes exist,
the only strategy considered in this thesis
is that of tokamak magnetic confinement,
briefly summarized in
Section~\ref{sec:Introduction:FusionEnergy:tokamaks}
While the tokamak's magnetic field's
prevent the hot, core plasma from contacting any material walls,
the plasma temperature and density
must drop to levels that are tolerable to the reactor walls,
located only a finite distance away from the plasma core;
the resulting gradients in density and temperature
provide free energy for numerous coherent and broadband instabilities,
some of which are measured and characterized in this thesis.


\subsection{The tokamak approach to fusion}
\label{sec:Introduction:FusionEnergy:tokamaks}

\begin{itemize}
  \item Doughnut shaped device characterized by a strong toroidal field
    and a large toroidal plasma current
  \item Most successful/promising configuration to date
  \item Numerous around the world, of all shapes and sizes
  \item Work reported in this thesis was conducted at \diiid~\cite{wesson}
\end{itemize}

The work reported in this thesis was conducted
at the \diiid\space tokamak in San Diego, CA~\cite[Sec.~12.5]{wesson}.
\diiid\space is a mid-size
($R = \SI{1.67}{\meter}$, $a = \SI{0.67}{\meter}$)
diverted tokamak with
maximum toroidal field $\SI{2.2}{\tesla}$ and
a maximum achieved plasma current of $\SI{3}{\mega\amp}$.
Up to $\SI{20}{\mega\watt}$ of deuterium neutral beam injection (NBI) and
$\SI{6}{\mega\watt}$ of electron cyclotron resonance heating (ECRH)
are available for auxiliary plasma heating.
The ECRH deposition location can be dynamically changed within a single shot,
and the NBI system can be operated in $\beta$ feedback
in an attempt to maintain constant plasma $\beta$
as other plasma parameters change.
\diiid\space is perhaps the best-diagnosed tokamak in the world,
with an extensive suite of equilibrium and fluctuation diagnostics.


\subsection{Fluctuations and transport in fusion plasmas}


\section{Interferometry}


\section{Thesis outline}
The remainder of this thesis is organized as follows:

\begin{itemize}
  \item Chapter~\ref{ch:InterferometricMethods} discusses
    the theory of optical interferometric methods
    in the context of measuring tokamak-plasma-density fluctuations.
    The laser-plasma interaction is quantified via
    Fraunhofer scalar-diffraction theory, and
    the resulting diffracted field is imaged on a square-law detector.
    Interfering the imaged field with a known reference field
    produces measurable intensity fluctuations;
    the specification of this reference field
    defines the interferometric method.
    Details of two particular interferometric methods ---
    external-reference-beam interferometry and
    phase contrast imaging (PCI) ---
    are discussed, with an emphasis on
    their sensitivity to fluctuations and their spatiotemporal bandwidths.
    Significantly, while PCI can measure fluctuations more sensitively
    than an external-reference-beam interferometer,
    PCI suffers from a low-$k$ cutoff;
    an external-reference-beam interferometer
    does \emph{not} suffer from such a low-$k$ cutoff.
  \item Chapter~\ref{ch:DesignConsiderations} considers the design of an
    external-reference-beam, heterodyne interferometer
    (hereafter referred to as a heterodyne interferometer).
    A criterion for satisfactory wavefront matching
    between the probe beam and the reference beam is developed, and
    finite-sampling-volume effects are shown
    to constrain the heterodyne interferometer's spatial bandwidth.
    The effects of phase noise, amplitude noise, and digitizer bit noise
    are each discussed in the context of
    the heterodyne interferometer's signal-to-noise ratio, and
    the systematic errors resulting from
    imperfect demodulation of the heterodyne interference signal
    are quantified.
  \item Chapter~\ref{ch:Implementation} details
    the addition of a heterodyne interferometer
    to the pre-existing PCI system on the \diiid\space tokamak;
    both systems operate simultaneously,
    sharing a single $\SI{10.6}{\micro\meter}$ probe beam through the plasma.
    Optical-diagnostic access on \diiid\space and the capabilities
    of the pre-existing PCI system are briefly reviewed.
    Referencing the design considerations
    in Chapter~\ref{ch:DesignConsiderations} and
    adopting the philosophy
    that the pre-existing PCI system should be minimally perturbed,
    the optical layout for the heterodyne interferometer is developed;
    the magnification of the interferometer's imaging system
    is selected such that the spatial bandwidths
    of the PCI and interferometer have a mid-$k$ overlap.
    The design, procurement, and installation
    of the new optical and electrical components
    required to make a heterodyne interferometric measurement
    are summarized.
    Of note is the interferometer's radio-frequency local oscillator:
    the phase noise of a crystal oscillator (XO)
    was empirically found to be \emph{too large}
    to make meaningful fluctuation measurements in most tokamak plasmas, but
    the substantially lower phase noise of an
    oven-controlled crystal oscillator (OCXO)
    allows measurements of a whole zoo
    of coherent and broadband fluctuations.
    The multiscale capabilities of the combined PCI-interferometer
    are empirically verified via sound-wave calibrations.
    Finally, some remarks are made regarding
    the use of such a combined PCI-interferometer
    on next-step devices, such as ITER.
  \item Chapter~\ref{ch:ToroidalCorrelation} discusses the correlation
    of the newly installed interferometer with
    \diiid's toroidally separated, pre-existing V2 interferometer.
    Capable of probing the core plasma,
    the interferometers are shown to be sensitive
    to core-localized fluctuations
    that are \emph{invisible} to external magnetic probes.
    The chapter begins with a brief review
    of the two-point correlation technique and
    shows how toroidal mode numbers can be extracted
    from a pair of toroidally separated measurements.
    Meaningful correlation requires that
    the two measurements share the same timebase.
    The digitizers of both interferometers were modified
    to phase lock their clocks, and
    a residual ``trigger offset'' was measured
    and is compensated in software.
    Where comparisons can be made with magnetic probes,
    the interferometer-measured toroidal mode numbers
    are in good agreement.
    Implications of the \SI{4}{\centi\meter} radial offset
    between the beam centers of each interferometer are discussed.
  \item Chapter~\ref{ch:TurbulenceMeasurements} demonstrates
    the ability of the combined PCI-interferometer
    to measure multiscale, broadband fluctuations.
    During a recent \diiid\space experiment,
    the location of electron-cyclotron heating (ECH)
    was moved from $\rho = 0.5 \rightarrow 0.8$,
    altering $T_e / T_i$
    in an attempt to change the coupling between
    the electron-scale and ion-scale turbulence.
    At low frequencies ($f < \SI{200}{\kilo\hertz}$),
    the low-$k$ and high-$k$ responses to the change in ECH location
    are similar;
    however, at higher frequencies, they are \emph{distinct}.
    Measurements are compared to linear predictions
    from the gyro-Landau-fluid code TGLF.
  \item Chapter~\ref{ch:Conclusions} concludes the thesis
    with a summary and discussion of the primary results
    and suggests avenues of future work.
\end{itemize}


\section{Units}
Unless explicitly stated, all formulas in this thesis are written in SI units.


\bibliographystyle{plainurl}
\bibliography{references}
