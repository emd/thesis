\chapter{Introduction}


\section{Fusion energy}
Fusion is the energy of the sun and stars, and,
for the better part of the last century,
man has been striving to replicate this process
in a controlled manner here on earth
for a source of safe, clean, virtually inexhaustible source of energy.


\section{Tokamak fusion}
\subsection{Turbulent transport}
\subsection{Energetic-particle confinement}


\section{Interferometry}


\section{Thesis outline}
The remainder of this thesis is organized as follows:

\begin{itemize}
  \item Chapter 2 discusses the theory of optical interferometric methods
    in the context of measuring tokamak-plasma-density fluctuations.
    The laser-plasma interaction is quantified via
    Fraunhofer scalar-diffraction theory, and
    the resulting diffracted field is imaged on a square-law detector.
    Interfering the imaged field with a known reference field
    produces measurable intensity fluctuations;
    the specification of this reference field
    defines the interferometric method.
    Details of two particular interferometric methods ---
    external-reference-beam interferometry and
    phase contrast imaging (PCI) ---
    are discussed, with an emphasis on
    their dynamic ranges and spatiotemporal bandwidths.
  \item Chapter 3
\end{itemize}


\section{Units}
