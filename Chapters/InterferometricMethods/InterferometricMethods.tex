\chapter{Interferometric methods in tokamak plasmas}


\section{Gaussian beam diffraction}


\subsection{Definition of a Gaussian beam}
A Gaussian beam of angular frequency $\omega_0$
propagating along the $z$-axis
has an electric field
\begin{equation}
  E(\vect{r}, t)
  =
  E(\vect{r}) e^{-i \omega_0 t}
\end{equation}
with spatial dependence~\cite{siegman_lasers}
\begin{equation}
  E(\vect{r})
  =
  E_0
  \frac{w_0}{w(z)}
  \exp\left[ \frac{-\rho^2}{w(z)^2} \right]
  \exp\left\{ i \left[
    k_0 z
    +
    \frac{k_0 \rho^2}{2 R(z)}
    -
    \psi(z) \right] \right\}
  \label{eq:InterferometricMethods:Gaussian_beam}
\end{equation}
Here,
$\rho = [x^2 + y^2]^{1/2}$ is the transverse distance from the optical axis,
$w_0$ is the width of the beam's waist, and
$k_0 = \omega_0 / c = 2 \pi / \lambda_0$ is the beam's wavenumber.
The beam's width $w(z)$, radius of curvature $R(z)$, and
Gouy phase $\psi(z)$ are defined as
\begin{align}
  w(z)
  &=
  w_0 \left[ 1 + \left( \frac{z}{z_R} \right)^2 \right]^{1/2}
  \label{eq:InterferometricMethods:Gaussian_beam_width}
  \\
  R(z)
  &=
  z \left[ 1 + \left( \frac{z_R}{z} \right)^2 \right]
  \label{eq:InterferometricMethods:Gaussian_beam_radius_of_curvature}
  \\
  \psi(z)
  &=
  \atan\left( \frac{z}{z_R} \right)
  \label{eq:InterferometricMethods:Gouy_phase}
\end{align}
where the Rayleigh range
\begin{equation}
  z_R \equiv \frac{\pi w_0^2}{\lambda_0}
  \label{eq:InterferometricMethods:Rayleigh_range}
\end{equation}
is the nominal division between the beam's
near-field ($|z| \ll z_R$) and far-field ($|z| \gg z_R$) behaviors.


\subsection{Kirchhoff diffraction theory}
\subsection{Fraunhofer diffraction of a free-space Gaussian beam}
\subsection{Fraunhofer diffraction of a phase-modulated Gaussian beam}



\section{Laser-plasma interactions in a tokamak}
\begin{itemize}
  \item Cold-plasma dispersion relation and derivation of $N$
  \item Expected phase shifts in tokamak plasma
  \item Refraction
  \item Scattering/diffraction
\end{itemize}

\section{Interferometry}
\begin{itemize}
  \item Homodyne
  \item Heterodyne
  \item Vibrations
  \item $k$-response
\end{itemize}

\section{Phase contrast imaging}
\begin{itemize}
  \item Principle
  \item $k$-response
\end{itemize}


\bibliographystyle{plainurl}
\bibliography{references}
