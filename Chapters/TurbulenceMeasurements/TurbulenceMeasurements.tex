\chapter{Multiscale turbulence measurements}
\label{ch:TurbulenceMeasurements}


\section{Doppler shift}
As a line-integrated measurement,
PCI is sensitive to fluctuations that are perpendicular
to the beam path ($z$-direction on \diiid).
Further, fluctuations are strongly field-aligned
such that they are approximately perpendicular
to the local magnetic field;
assuming the fluctuations are electrostatic,
the local magnetic field is well-represented
by the equilibrium field $\vect{B}_0$.
Thus, fluctuations measured by PCI have wavevectors satisfying
\begin{equation}
  \vect{k} = |k| (\hat{z} \cross \vect{B}_0)
\end{equation}
In general, then, the wavevector has major-radial and toroidal components.

The Doppler shift is $\vect{k} \cdot \vect{v}$,
where $\vect{v}$ is approximately a flux function.
Thus, both the toroidal component of the velocity and
the major-radial projection of the poloidal velocity
contribute to the PCI-measured Doppler shift.
Thus, I should compare the PCI-measured phase velocity
to the vector sum of (a) the toroidal velocity and
(b) the major-radial projection of the poloidal velocity.

In general, in the lab frame, the velocity perpendicular to the field line
can be expressed as the fluid velocity plus the advection
of the plasma as a rigid body, i.e.\ the $\vect{E} \cross \vect{B}$.
Therefore, the measured frequency is given by
$\vect{k} \cdot (\vect{v}_{\text{ph}} + \vect{v}_{\vect{E} \times \vect{B}})$.
Assuming that
$|\vect{v}_{\text{ph}}| \ll \vect{v}_{\vect{E} \times \vect{B}}|$,
which still needs to be verified in every case,
you can localize the measurement by matching the observed frequency
to that due to the Doppler shift induced
by the rigid $\vect{E} \times \vect{B}$ rotation.
% by the rigid $E \times B$ rotation.
Unfortunately it doesn't work every time\ldots
