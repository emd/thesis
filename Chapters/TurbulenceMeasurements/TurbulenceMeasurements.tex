\chapter{Multiscale turbulence measurements}
\label{ch:TurbulenceMeasurements}
Development of a first-principles understanding
of turbulent transport in a tokamak
requires multi-tiered validation\ldots
comparing not just heat fluxes, but
also turbulent spectra etc.
\diiid's extensive suite of fluctuation diagnostics
provides an ideal setting to validate
predicted changes~\cite{howard_pp16}
to turbulent spectra when altering relative drive
between electron-scale and ion-scale turbulence.


\section{Overview of multiscale gyrokinetic predictions}


\section{Experimental conditions}
The experiment was run in the ITER-similar shape,
with aspect ratio, elongation, and triangularity
all closely matched to those of the ITER-baseline scenario
\cite[Sec.~13.5 \& 13.6]{wesson}.
The on-axis toroidal field $B_T = \SI{1.7}{\tesla}$ and
plasma current $I_p = \SI{1.3}{\mega\ampere}$
produced $q_{95} = 3.15$,
where $q_{95}$ is the average value
of the safety factor $q$~\cite[Sec.~3.4]{wesson}
over the surface that encloses $95\%$
of the poloidal flux within the last-closed flux surface.
The neutral beams~\cite[Sec.~5.3-5.5]{wesson}
were operated with feedback to maintain
$\beta_N = 1.9$, where
$\beta_N$ is the normalized plasma pressure~\cite[Sec.~6.18]{wesson}.
In order to suppress core MHD,
an average neutral-beam torque of $\sim \SI{1.5}{\newton \meter}$
was injected into the plasma;
note that this is $\sim 4\times$ larger than
the projected ITER-equivalent torque~\cite{garofalo_nf11}.
In order to alter the local electron-scale and ion-scale drives,
the electron cyclotron resonance heating (ECH)~\cite[Sec.~5.10]{wesson}
location was scanned between $\rho = 0.5$ and $\rho = 0.8$,
where $\rho$ is the square root of the normalized toroidal flux
(which scales as $r / a$, with
$r$ being the minor-radial coordinate and
$a$ being the minor radius of the plasma).
Intra-shot scans of the ECH location
were plagued with core MHD, so
only shot-to-shot, MHD-free scans of the ECH location
are considered here.
The line-averaged density was
$\bar{n}_e = \SI{5.2e19}{\per\meter\cubed}$.
Impurities are removed from the plasma
by both large and small edge localized modes (ELMs)~\cite[Sec.~7.17]{wesson}.
The time histories of several actuators and plasma parameters
are shown in Figure~\ref{fig:TurbulenceMeasurements:traces}.
Note that multiscale gyrokinetic simulations
of this experiment's reference discharge,
\diiid\space shot $153523$ with ECH at $\rho = 0.5$,
indicate that the turbulent transport
is intrinsically multiscale in nature~\cite{holland_nf17}.

\begin{figure}
  \centering
  \includegraphics[width = \textwidth]{%
    Chapters/TurbulenceMeasurements/figs/traces.png}
  \caption[Time histories of various actuators \& plasma parameters]{%
    Time histories of various actuators and plasma parameters:
    (a) plasma current $I_p$,
    (b) neutral beam injected (NBI) power $P_{\text{inj}}$,
    (c) NBI torque $T_{\text{inj}}$,
    (d) electron cyclotron resonance heating (ECH) power $P_{\text{ECH}}$,
    (e) line-averaged density $\bar{n}_e$,
    (f) normalized plasma pressure $\beta_N$,
    (g) confinement quality $H_{98,\text{y}2}$, and
    (h) divertor $D_{\alpha}$ light, indicating
    the presence of large and small edge localized modes (ELMs).
    The ECH heating location was
    at $\rho = 0.5$ in $171536$ and
    at $\rho = 0.8$ in $171538$.
  }
\label{fig:TurbulenceMeasurements:traces}
\end{figure}

Equilibrium profiles
\begin{itemize}
  \item averaging over \textcolor{red}{XXX ms}
  \item What types of fits? (splines, RBF, etc.)
  \item Electron-density profiles -- Thomson w/ CO$_2$ normalization?
    Reflectometer?
  \item Electron-temperature profiles -- Thomson \& ECE (or ECE cutoff?)
  \item Ion-temperature profiles -- \textcolor{red}{???}
  \item CER provides C$^{6+}$ density, temperature, and rotation
  \item $E_r$ computed using force balance for
    C$^{6+}$ pressure and toroidal rotation.
    \textcolor{red}{poloidal rotation neglected?}
  \item \textcolor{red}{MSE-constrained EFIT? Kinetic?}
  \item \textcolor{red}{How are ELMs handled?}
  \item Uncertainties
  \item Clear change in turbulent drives with ECH location
  \item The profiles are shown in
    Figure~\ref{fig:TurbulenceMeasurements:profiles}.
\end{itemize}

\begin{figure}
  \centering
  \includegraphics[width = \textwidth]{%
    Chapters/TurbulenceMeasurements/figs/profiles.pdf}
  \caption[Equilibrium profiles, inverse scale lengths, \& $\ExB$ shearing rate]{%
    Profiles, inverse scale lengths, and $\ExB$ shearing rate:
    (a) electron density $n_e$,
    (b) electron temperature $T_e$,
    (c) deuterium temperature $T_i$,
    (d) radial electric field $E_r$,
    (e) normalized inverse $n_e$ scale length $a / L_{n_e}$,
    (f) normalized inverse $T_e$ scale length $a / L_{T_e}$,
    (g) normalized inverse $T_i$ scale length $a / L_{T_i}$,
    (h) $\ExB$ shearing rate $\gamma_E$.
    The ECH heating location was
    at $\rho = 0.5$ in $171536$ and
    at $\rho = 0.8$ in $171538$.
  }
\label{fig:TurbulenceMeasurements:profiles}
\end{figure}


\section{Measurements}


\subsection{ELM filtering}
Edge localized modes (ELMs) expel impurities from the plasma but
will also present severe challenges to plasma-facing components
in future reactors~\cite[Sec.~7.17]{wesson}.
Because of their virulence and their bursty nature,
ELMs produce strong spiking in the interferometer and PCI measurements,
whitening the measured spectra
[Sec.~10.3.2.3]\cite{bendat_and_piersol}.
Additionally, the temperature and density profiles relax during an ELM,
altering the turbulent drives in the plasma edge.
In order to accurately estimate
the spectrum of the background turbulence, then,
the ELM contributions to the interferometer and PCI measurements
must be removed.

\begin{figure}
  \centering
  \includegraphics[width = \textwidth]{%
    Chapters/TurbulenceMeasurements/figs/ELM_filtering_example.png}
  \caption[ELM filtering]{%
    Edge localized modes (ELMs) must be removed
    from the PCI and interferometer measurements
    prior to spectral analysis of the background turbulence.
    (Upper panel): The interferometer-measured
    fluctuating phase $\tilde{\phi}$,
    with large, ELM-induced spiking.
    (Lower panel): Divertor $D_{\alpha}$ emission,
    indicating the presence of large Type I ELMs
    as well as smaller ELMs.
    Windows \emph{excluded} from spectral analysis are shown in gray.
    The \diiid\space shot number is shown in the upper right
    of the lower panel.
  }
\label{fig:TurbulenceMeasurements:ELM_filtering_example}
\end{figure}

In this work, ELMs are simply and automatically detected
using measurements from the interferometer.
After the high-pass filtering described in
Section~\ref{sec:Implementation:DataPreparation:high_pass_filtering},
the interferometer-measured fluctuating phase $\tilde{\phi}$
is a zero-mean, random process,
as shown in the upper panel of
Figure~\ref{fig:TurbulenceMeasurements:ELM_filtering_example}.
Large, intermittent spikes pepper $\tilde{\phi}(t)$ during ELMy H-mode, and
the lower panel of
Figure~\ref{fig:TurbulenceMeasurements:ELM_filtering_example}
indicates that these spikes are well correlated
with ELM-induced $D_{\alpha}$ emission in the divertor.
While the $D_{\alpha}$ emission following large Type I ELMs
exhibits a relatively slow decay,
the interferometer-measured $\tilde{\phi}$
returns to stationarity much more rapidly.
Thus, it is desirable to identify
stationary inter-ELM windows
from the interferometer measurements
rather than the $D_{\alpha}$ emission.
Points in the interferometer-measured $\tilde{\phi}$
exceeding $3 \times$ the RMS value
are identified as ELMs, and
successive ELMs are required to be separated
by at least a $\SI{0.5}{\milli\second}$ ``debouncing time''
(spikes separated by less than the debouncing time
are classified as belonging to the same ELM).
Subsequent spectral analysis is then performed
using only the $20\%$ -- $80\%$ inter-ELM windows
of the interferometer and PCI measurements.
Figure~\ref{fig:TurbulenceMeasurements:ELM_filtering_example}
shows the windows \emph{excluded} from spectral analysis in gray.


\subsection{Frequency spectra}
\begin{figure}
  \centering
  \includegraphics[width = \textwidth]{%
    Chapters/TurbulenceMeasurements/figs/Sf_interferometer_pci.pdf}
  \caption[Interferometer and PCI frequency spectra]{%
    Interferometer and PCI frequency spectra.
  }
\label{fig:TurbulenceMeasurements:Sf_interferometer_pci}
\end{figure}


\subsection{Frequency-wavenumber spectra}
\begin{itemize}
  \item Complex-correlation function
  \item $S(k,f)$
\end{itemize}

\begin{figure}
  \centering
  \includegraphics[width = \textwidth]{%
    Chapters/TurbulenceMeasurements/figs/Skf_pci.png}
  \caption[PCI frequency-wavenumber spectra]{%
    PCI frequency-wavenumber spectra.
  }
\label{fig:TurbulenceMeasurements:Skf_pci}
\end{figure}


\subsection{Wavenumber spectra}
\begin{figure}
  \centering
  \includegraphics[width = 0.6 \textwidth]{%
    Chapters/TurbulenceMeasurements/figs/Sk_power_law.pdf}
  \caption[PCI-measured power law]{%
    PCI-measured power law from branches annotated in
    Figure~\ref{fig:TurbulenceMeasurements:Skf_annotated}.
  }
\label{fig:TurbulenceMeasurements:Sk_power_law}
\end{figure}


\section{TGLF modeling}
\begin{figure}[h!]
  \centering
  \includegraphics[width = \textwidth]{%
    Chapters/TurbulenceMeasurements/figs/doppler_shift.pdf}
  \caption[Doppler shift]{%
    Doppler shift. Positive \& negative values of $v_{\text{pci}}$
    correspond to above \& below the midplane, respectively.
    As the mask was not used, we cannot determine whether or not
    $v_{\text{meas}}$ corresponds to above or below the midplane, however
    (hence we plot $\pm v_{\text{meas}}$).
  }
\label{fig:TurbulenceMeasurements:doppler_shift}
\end{figure}

\begin{figure}[h!]
  \centering
  \includegraphics[width = \textwidth]{%
    Chapters/TurbulenceMeasurements/figs/TGLF_171536_vs_171538.png}
  \caption[Qualitative presentation of TGLF results]{%
    Qualitative presentation of TGLF results.
  }
\label{fig:TurbulenceMeasurements:TGLF_171536_vs_171538}
\end{figure}

\begin{figure}[h!]
  \centering
  \includegraphics[width = \textwidth]{%
    Chapters/TurbulenceMeasurements/figs/linear_stability.pdf}
  \caption[Growth rates \& frequencies at $\rho=0.7$]{%
    Growth rates \& frequencies at $\rho=0.7$
    (i.e.\ we're slicing
    Figure~\ref{fig:TurbulenceMeasurements:TGLF_171536_vs_171538}
    at $\rho = 0.7$ for the most unstable mode, mode $1$).
    Note that only $171536$ uses SAT\_RULE $= 1$,
    which is calibrated against multiscale GYRO results;
    $171538$ uses SAT\_RULE $= 0$.
    Thus, it is difficult to draw conclusions\ldots
  }
\label{fig:TurbulenceMeasurements:linear_stability}
\end{figure}

\begin{figure}[h!]
  \centering
  \includegraphics[width = \textwidth]{%
    Chapters/TurbulenceMeasurements/figs/density_spectra.pdf}
  \caption[TGLF-predicted density-fluctuation spectra at $\rho=0.7$]{%
    TGLF-predicted density-fluctuation spectra at $\rho=0.7$.
    Note that only $171536$ uses SAT\_RULE $= 1$,
    which is calibrated against multiscale GYRO results;
    $171538$ uses SAT\_RULE $= 0$.
    Thus, it is difficult to draw conclusions\ldots
  }
\label{fig:TurbulenceMeasurements:density_spectra}
\end{figure}


\bibliographystyle{plainurl}
\bibliography{references}
