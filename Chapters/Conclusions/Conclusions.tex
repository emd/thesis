\chapter{Conclusions \& future work}
\label{ch:Conclusions}


\section{Summary \& conclusions}
\label{sec:Conclusions:summary_and_conclusions}
The work described in this thesis can be summarized as follows:
\begin{itemize}
  \item Chapter~\ref{ch:InterferometricMethods} discusses
    the theory of optical interferometric methods
    in the context of measuring tokamak plasma-density fluctuations.
    The laser-plasma interaction is quantified via
    Fraunhofer scalar-diffraction theory, and
    the resulting diffracted field is imaged onto a square-law detector.
    Interfering the imaged field with a known reference field
    produces measurable intensity fluctuations;
    the specification of this reference field
    defines the interferometric method.
    Details of two particular interferometric methods ---
    external-reference-beam interferometry and
    phase contrast imaging (PCI) ---
    are discussed, with an emphasis on
    their sensitivity to fluctuations and their spatiotemporal bandwidths.
    Significantly, while PCI can measure fluctuations more sensitively
    than an external-reference-beam interferometer,
    PCI suffers from a low-$k$ cutoff;
    an external-reference-beam interferometer
    does \emph{not} suffer from such a low-$k$ cutoff.
  \item Chapter~\ref{ch:DesignConsiderations} considers the design of an
    external-reference-beam, heterodyne interferometer
    (hereafter referred to as a heterodyne interferometer).
    A criterion for satisfactory wavefront matching
    between the probe beam and the reference beam is developed, and
    finite-sampling-volume effects are shown
    to constrain the heterodyne interferometer's spatial bandwidth.
    The effects of phase noise, amplitude noise, and digitizer bit noise
    are each discussed in the context of
    the heterodyne interferometer's signal-to-noise ratio, and
    the systematic errors resulting from
    imperfect demodulation of the heterodyne interference signal
    are quantified.
  \item Chapter~\ref{ch:Implementation} details
    the addition of a heterodyne interferometer
    to the pre-existing PCI system on the \diiid\space tokamak.
    Both systems operate simultaneously,
    sharing a single $\SI{10.6}{\micro\meter}$ probe beam through the plasma.
    Optical-diagnostic access on \diiid\space and the capabilities
    of the pre-existing PCI system are briefly reviewed.
    Referencing the design considerations
    in Chapter~\ref{ch:DesignConsiderations} and
    adopting the philosophy
    that the pre-existing PCI system should be minimally perturbed,
    the optical layout for the heterodyne interferometer is developed;
    the magnification of the interferometer's imaging system
    is selected such that the spatial bandwidths
    of the PCI and interferometer have a mid-$k$ overlap.
    The design, procurement, and installation
    of the new optical and electrical components
    required to make the heterodyne interferometric measurement
    are summarized.
    Of note is the interferometer's radio-frequency local oscillator:
    the phase noise of a crystal oscillator (XO)
    was empirically found to be \emph{too large}
    to make meaningful fluctuation measurements in most tokamak plasmas, but
    the substantially lower phase noise of an
    oven-controlled crystal oscillator (OCXO)
    allows measurements of a whole zoo
    of coherent and broadband plasma fluctuations.
    The interferometer response and
    the multiscale capabilities of the combined PCI-interferometer
    are empirically verified via sound-wave calibrations.
    Specifically, the PCI is shown to measure high-$k$
    ($\SI{1.5}{\per\centi\meter} < |k_R| \leq \SI{25}{\per\centi\meter}$)
    fluctuations with
    sensitivity $3 \times 10^{13} \; \text{m}^{-2} / \sqrt{\text{kHz}}$,
    while the interferometer simultaneously measures low-$k$
    ($|k_R| < \SI{5}{\per\centi\meter}$) fluctuations with
    sensitivity $3 \times 10^{14} \; \text{m}^{-2} / \sqrt{\text{kHz}}$.
    Both systems have temporal bandwidths in excess of $\SI{1}{\mega\hertz}$.
  \item Chapter~\ref{ch:ToroidalCorrelation} discusses the correlation
    of the newly installed interferometer with
    \diiid's toroidally separated, pre-existing V$2$ interferometer.
    Capable of probing the core plasma,
    the interferometers are shown to be sensitive
    to core-localized fluctuations
    that are \emph{invisible} to external magnetic probes.
    The chapter begins with a brief review
    of the two-point correlation technique and
    shows how toroidal mode numbers can be extracted
    from a pair of toroidally separated measurements.
    Meaningful correlation requires that
    the two measurements share the same timebase.
    The digitizers of both interferometers were modified
    to phase lock their clocks, and
    a residual ``trigger offset'' was measured
    and is compensated in software.
    Where comparisons can be made with magnetic probes,
    the interferometer-measured toroidal mode numbers
    are in good agreement.
    Currently, there is not a tested, robust method
    for correcting the bias introduced by
    the $\SI{4}{\centi\meter}$ major-radial offset
    between the interferometer beam centers,
    which unfortunately limits the
    deployment of this system for physics studies
    of core-localized MHD.
  \item Chapter~\ref{ch:TurbulenceMeasurements} demonstrates
    the multiscale capabilities of the combined PCI-interferometer.
    During a recent \diiid\space experiment,
    the location of electron cyclotron resonance heating (ECH)
    was moved from $\rhoech = 0.5$ to $\rho_{ECH} = 0.8$,
    altering the local $a / L_{T_e}$ and $a / L_{T_i}$
    in an attempt to change the coupling between
    the electron-scale and ion-scale turbulence.
    As such, this experiment presents an ideal opportunity
    for multiscale turbulence investigations
    with the combined PCI-interferometer.
    Numerous turbulent branches are observed.
    In particular, the interferometer measures
    a low-$k$ electromagnetic mode driven unstable by collisionality,
    properties consistent with the micro-tearing mode (MTM), and
    the PCI measures a wavenumber spectrum
    that exhibits distinct flattening
    when $a / L_{T_e}$ is increased relative to $a / L_{T_i}$,
    reminiscent of results
    from realistic multiscale gyrokinetic simulations~\cite{howard_pp16}.
    To aid the interpretation of these measurements,
    linear-stability analysis and quasilinear-transport modeling
    are performed with the gyro-Landau fluid code TGLF, and
    qualitative agreement with the PCI-measured wavenumber spectrum
    is obtained.
\end{itemize}


\section{Future work}
\label{sec:Conclusions:future_work}
The combined PCI-interferometer developed in this work
has a clear application in the burgeoning study
of multiscale turbulence and cross-scale coupling, which
may be significant in the reactor relevant $T_e \approx T_i$ regime.
In roughly the next six months,
Howard \emph{et al.} expects to complete
realistic multiscale gyrokinetic simulations
for the experiment described in
Chapter~\ref{ch:TurbulenceMeasurements}.
It will be very interesting
to see if the predicted wavenumber spectrum
matches the PCI-measured wavenumber spectrum.
It should be noted that a synthetic PCI diagnostic
already exists for the interpretation
of such gyrokinetic simulations~\cite{rost_pp10}.
Small modifications to the synthetic PCI
should also allow a synthetic interferometer diagnostic.
Previous multiscale simulations predict
significant local and non-local energy cascades
between the ion and electron scales~\cite{howard_pp16}, so
it is desirable to investigate such coupling empirically.
With its large spatiotemporal bandwidth,
the combined PCI-interferometer may be ideally suited
for measurement of such coupling, which
may be suitably quantified by
the bicoherence~\cite{young_and_powers_ieee79}
between various channels of the system or
some other suitable measure of nonlinear processes.
(Note that the author has performed preliminary bispectral analysis
of the measurements discussed in
Chapter~\ref{ch:TurbulenceMeasurements};
interestingly, the $|k_R| \sim \SI{5}{\per\centi\meter}$ and
$f \sim \SI{1}{\mega\hertz}$ mode observed in
Figure~\ref{fig:TurbulenceMeasurements:Skf_pci}(b)
has an exceptionally large autobicoherence).

The interferometer-measured, low-$k$, electromagnetic modes
that are destabilized by collisionality
are also deserving of further study.
The properties of this mode are consistent
with the micro-tearing mode, which
was predicted to be marginally unstable
in the multiscale experiment's reference discharge~\cite{holland_nf17}.
Unfortunately, TGLF's default eigenfunction basis
of four Hermite polynomials is typically insufficient
to resolve micro-tearing modes (MTMs)~\cite{staebler_MTM_question}, so
there are no attempts to simulate the MTM in this work.
However, it may be conceivable that
increasing the number of Hermite polynomials
will allow identification of the MTM in TGLF.
Alternatively, linear simulations with
the gyrokinetic code GYRO~\cite{candy_jcp03}
could be pursued.
(Note that the reference-discharge simulations
indicating marginal MTM instability were performed with GYRO).
Experimentally, it is desirable to map out the parametric dependencies
of this mode, particularly its response
to the plasma $\beta$ and collisionality.
If dedicated experiments cannot be performed,
it should be noted that the relevant experimental conditions
(i.e.\ ITER-baseline scenario) are fairly typical at \diiid, and
a fair amount may still be learned
by ``piggybacking'' on other experiments.

With regards to the combined PCI-interferometer,
the most substantial improvement to the system
would be upgrading the heterodyne-interferometer detector
from a single element to a multi-element array.
This would allow reconstruction of $k_R$
from the interferometer measurements,
enabling estimates of
frequency-wavenumber spectra $S_{\phi,\phi}(k,f)$ and
wavenumber spectra $S_{\phi,\phi}(k)$
much like with the PCI.
This capability is desirable for several reasons.
First and foremost,
interferometric measurements across a multi-element array
would allow accurate estimates of $S_{\phi,\phi}(k)$
below the PCI low-$k$ cutoff
(\ref{eq:Implementation:kg_realized}), which
may have important implications for
validation of spectral-flattening predictions
from multiscale gyrokinetic predictions.
Further, as discussed in
Section~\ref{sec:Implementation:Calibration:pci},
interferometric measurements across a multi-element array
would allow robust and accurate
cross-calibration of the PCI
on a shot-to-shot and an intra-shot basis.
Note that each additional detector element
would require its own set of electronics
(e.g.\ signal conditioning RF amplifiers,
demodulation electronics, and
audio amplifiers) and
two additional digitizer channels
(to digitize both the in-phase $I$ and quadrature $Q$ signals).
While the ``deadbug'' circuit construction utilized in this work
is ideal for prototyping,
any future increase to the number of interferometer channels
would call for a printed-circuit-board (PCB) construction
of the electronics.
Thus, increasing the number of interferometer channels
is not a small undertaking.

A simpler, cheaper, and faster performance improvement
would be the procurement of anti-aliasing filters
with a higher cutoff frequency.
The current anti-aliasing filters
limit the bandwidth of the interferometer to
approximately $\SI{1}{\mega\hertz}$, but
the upstream components have bandwidths
in excess of $\SI{2}{\mega\hertz}$.
Thus, new anti-aliasing filters could,
quite literally overnight,
nearly double the temporal bandwidth of the interferometer.

Finally, there is not currently a tested, robust method
for correcting the bias introduced by
the major-radial offset of the toroidally correlated interferometers
(other than reducing the offset, i.e.\ $\Delta R \rightarrow 0$,
which is not possible with current port allocations).
It may be possible to account for the radial and poloidal mode structure
via e.g.\ measurements from
microwave imaging reflectometry (MIR)~\cite{muscatello_rsi14} or
electron cyclotron emission imaging (ECEI)~\cite{tobias_rsi10}.

Looking towards ITER and other next-step devices,
the combined PCI-interferometer may allow
sensitive, high spatiotemporal bandwidth measurements
of multiscale turbulence.
The diagnostic development pursued in this thesis
proves that heterodyne-interferometric detection and PCI detection
can be simultaneously implemented
using a shared probe beam and
a shared set of ports.
The addition of PCI detection to
e.g.\ the ITER interferometer~\cite{vanzeeland_TIP_rsi13}, however,
is not without its challenges.
For example, the ITER interferometer
employs a Michelson configuration,
with the beam making a second pass through the plasma
after bouncing off of a retroreflector inside the vacuum vessel.
In contrast,
at least to the author's knowledge,
all previous PCI implementations
have employed a Mach-Zehnder configuration,
with the probe beam making a single pass through the plasma.
In principle, PCI can use a Michelson configuration, but
the double pass and retroreflector
may complicate interpretation of the measurements,
particularly if attempting to localize the measurements
with a spatially filtering mask~\cite{dorris_rsi09, dorris_phd, lin_rsi06} or
with $2$-dimensional detector arrays~\cite{sanin_rsi04, tanaka_rsi16}.

Regardless, the spatial bandwidth of a PCI system
that shares its probe beam
with the ITER interferometer can be considered.
The 1/e $E$ waist of the ITER interferometer's
$\SI{10.6}{\micro\meter}$ probe beam is
$w_0 \approx \SI{8}{\milli\meter}$~\cite{vanzeeland_TIP_rsi13}
such that a PCI system using this probe beam
would have a diffraction-limited low-$k$ cutoff
(\ref{eq:InterferometricMethods:pci_kmin_physics})
of $\SI{2.5}{\per\centi\meter}$.
(However, recall that the \diiid\space PCI
is operated two to three times above the diffraction limit
to give some leeway to the PCI feedback system).
Of course, as demonstrated in this thesis,
simultaneous heterodyne-interferometric and PCI detection
can obviate the PCI's low-$k$ cutoff.
A somewhat larger problem, however, may be the limited collection volumes
and long path lengths
between the vacuum vessel and the detector, which
may impose severe constraints
on the high-$k$ cutoff of a $\SI{10.6}{\micro\meter}$ PCI or interferometer.
One potential solution is to use a smaller probe wavelength
(i.e.\ larger $k_0$) to decrease the scattering angle $\theta_m$ from
(\ref{eq:GaussianBeamDiffraction:scattering_angles}).
As the burning plasma regime will be predominantly electron heated
(i.e.\ via fusion alpha particles slowing down on electrons),
it is extremely important that
both high-$k$ electron turbulence and
any cross-scale coupling with low-$k$ ion turbulence
is accurately diagnosed and understood.


\bibliographystyle{plainurl}
\bibliography{references}
