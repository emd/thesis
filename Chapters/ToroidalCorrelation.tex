\chapter{Correlation of \diiid's toroidally separated interferometers}


\section{Perturbed plasma density}
An MHD mode displaces a plasma from it's equilibrium position
by $\vect{\xi} = \vect{\xi}(\vect{r}, t)$.
The perturbed velocity is given as
$\vect{v_1} = \partial \vect{\xi} / \partial t$
and, assuming harmonic variations, reduces to
\begin{equation}
  \vect{v_1}
  \equiv
  \frac{\partial \vect{\xi}}{\partial t}
  =
  -i \omega \vect{\xi}
  \notag
\end{equation}
where $\omega$ is the mode's angular frequency.
The plasma density $n_i$ is given as
\begin{equation}
  n_i = \bar{n}_i + \tilde{n}_i
  \notag
\end{equation}
where $\bar{n}_i$ and $\tilde{n}_i$ are
the equilibrium and fluctuating components, respectively.
Assuming a stationary equilibrium ($\vect{v_0} = 0$) and
using the above relations,
the linearized continuity equation reduces to
\begin{equation}
  \tilde{n}_i = -\nabla \cdot (\bar{n}_i \vect{\xi})
  \label{eq:ToroidalCorrelation:density_fluctuations}
\end{equation}
If we relax the assumption on $\vect{v_0}$ to allow
finite equilibrium flow ($\vect{v_0} \neq 0$),
then the right-hand side of (\ref{eq:ToroidalCorrelation:density_fluctuations})
is simply multiplied by the prefactor
$[1
- (\vect{v_0} \cdot \vect{k} / \omega)
+ i (\nabla \cdot \vect{v_0} / \omega)]^{-1}$, where
$\vect{k}$ is the mode wavevector.


\section{Interferometer phase fluctuations}
For a CO$_2$ laser beam ($\lambda_0 =$ \SI{10.6}{\micro \meter})
in a tokamak plasma, the index of refraction $N$ is
\begin{equation}
  N
  \approx
  1 - \frac{1}{2} \left( \frac{\omega_{pe}}{\omega_0} \right)^2
  \notag
\end{equation}
where $\omega_{pe}$ is the electron angular plasma frequency and
$\omega_0 = 2 \pi c / \lambda_0 = 2 \pi \cdot \SI{28.3}{\tera\hertz}$
is the laser's angular frequency.
Thus, a CO$_2$ beam propagating through a tokamak plasma
will acquire a phase shift $\phi$ \emph{relative} to vacuum
\begin{equation}
    \phi
    =
    \frac{\omega}{c} \int (N - 1) dl
    =
    - r_e \lambda_0 \int n_e dl \notag
\end{equation}
where $r_e = \SI{2.8e-15}{\meter}$ is the classical electron radius.
Further, if there are electron density fluctuations $\tilde{n}_e$
about the equilibrium $\bar{n}_e$, there will be corresponding
phase fluctuations $\tilde{\phi}$
\begin{align}
  \tilde{\phi} = -r_e \lambda_0 \int \tilde{n}_e dl
  \label{eq:ToroidalCorrelation:phase_fluctuations_generic}
\end{align}
Assuming quasineutrality $n_e \approx n_i$ and
invoking (\ref{eq:ToroidalCorrelation:density_fluctuations}),
the phase fluctuations reduce to
\begin{align}
  \tilde{\phi}
  =
  r_e \lambda_0
  \int [\nabla \cdot (\bar{n}_e \vect{\xi})] dl
  \label{eq:ToroidalCorrelation:phase_fluctuations_from_displacement}
\end{align}

Now, \diiid's V2 and PCI interferometers have \emph{vertical} beam paths
located at major radial ($R$) and toroidal ($\zeta$) coordinates
$(R_1, \zeta_1) = (\SI{1.94}{\meter}, \; 240^{\circ})$ and
$(R_2, \zeta_2) = (\SI{1.98}{\meter}, \; 285^{\circ})$, respectively.
Fourier decomposing the displacement $\vect{\xi}$ as
\begin{equation}
  \vect{\xi}(\vect{r}, t)
  =
  \vect{\xi_0}(r) e^{i(m \theta + n \zeta - \omega t)}
  \notag
\end{equation}
for poloidal ($m$) and toroidal ($n$) mode numbers and
$\vect{\xi_0}(r) \in \mathbb{R}^3$,
(\ref{eq:ToroidalCorrelation:phase_fluctuations_from_displacement}) becomes
\begin{align}
  \tilde{\phi}
  =
  r_e \lambda_0
  \int \left\{
    \nabla
    \cdot
    \left[
      \bar{n}_e \, \vect{\xi_0}(r) e^{i(m \theta + n \zeta - \omega t)}
    \right]
  \right\} dl
  \notag
\end{align}
Finally, noting that $dl = dl(r, \theta)$ for vertical beam paths,
the phase fluctuations reduce to
\begin{equation}
  \tilde{\phi}
  =
  \Phi e^{i(n \zeta - \omega t)}
  \label{eq:ToroidalCorrelation:phase_fluctuations_vertical_beam1}
\end{equation}
where
$\Phi
\equiv
\Phi(R, m, n, \vect{\xi_0}(r), \bar{n}_e(r), \vect{G}) \in \mathbb{C}$
is a complex-valued function of
the beam's major radial location,
the mode structure,
the equilibrium density profile, and
the plasma geometry $\vect{G} = \vect{G}(R_0, a, \kappa, \delta, \cdots)$.
$\Phi$ can be written explicitly
as a complex value $\Phi = |\Phi| e^{i \sigma}$.

For a given mode and plasma,
the V2 and PCI interferometer beams see the \emph{same}
$\{m, n, \vect{\xi_0}(r), \bar{n}_e(r), \vect{G}\}$, and
$\Phi$ reduces to a one-dimensional function $\Phi = \Phi(R)$.
Thus, (\ref{eq:ToroidalCorrelation:phase_fluctuations_vertical_beam1}) can
alternatively be written as
\begin{equation}
  \tilde{\phi}
  =
  |\Phi(R)| e^{i[n \zeta - \omega t + \sigma(R)]}
  \label{eq:ToroidalCorrelation:phase_fluctuations_vertical_beam2}
\end{equation}
where the dependence on $R$ for a given mode and plasma
has been noted explicitly.
The phase angle $\alpha$ of $\tilde{\phi}$ is defined as
\begin{equation}
  \alpha \equiv n \zeta - \omega t + \sigma(R)
  \label{eq:ToroidalCorrelation:phase_angle}
\end{equation}
such that $\tilde{\phi} = |\Phi| e^{i \alpha}$.


\section{Toroidal correlations --- the ideal case}
In the ideal case, phase fluctuations $\tilde{\phi}_1$ and $\tilde{\phi}_2$
are made at different toroidal locations $\zeta_1 \neq \zeta_2$ but
the \emph{same} radial locations $R_1 = R_2 = R$.
The one-sided cross-spectral density
$G_{12}(f) = |G_{12}(f)| e^{i \alpha_{12}(f)}$
then yields an estimate of the relative phase angle
$\alpha_{12} \equiv \alpha_2 - \alpha_1 = n(\zeta_2 - \zeta_1)$,
inspiring the definition of the measured toroidal mode number as
\begin{equation}
  n_{\text{meas}}
  \equiv
  \frac{\alpha_{12}}{\Delta \zeta}
  \quad \text{where} \quad
  \Delta \zeta \equiv \zeta_2 - \zeta_1
  \label{eq:ToroidalCorrelation:toroidal_mode_number_ideal}
\end{equation}
