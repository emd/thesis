\chapter{Sound-wave characterization}
\label{app:SoundWaveCharacterization}
\begin{itemize}
  \item Sound-wave dispersion relation
\end{itemize}


\section{Hardware}
\label{sec:SoundWaveCharacterization:Hardware}
\subsection{Speaker}
\label{sec:SoundWaveCharacterization:Hardware:speaker}
\subsection{Calibrated microphone}
\label{sec:SoundWaveCharacterization:Hardware:microphone}
\subsection{Test stand}
\label{sec:SoundWaveCharacterization:Hardware:test_stand}


\section{Sound-wave measurements}
\label{sec:SoundWaveCharacterization:Measurements}
To lowest order, the speaker is cylindrically symmetric.
Thus, the sound waves are expected to have
axial, radial, and frequency dependencies.
Sections~\ref{sec:SoundWaveCharacterization:Measurements:amlitude} through
\ref{sec:SoundWaveCharacterization:Measurements:phasing}
summarize these measurements and their implications
for the sound-wave model developed in
Section~\ref{sec:SoundWaveCharacterization:Model}.


\subsection{On-axis amplitude}
\label{sec:SoundWaveCharacterization:Measurements:amlitude}
After centering the microphone on the speaker's symmetry axis,
the on-axis amplitude can be easily characterized by
varying both the frequency $f$ of the sound waves and
the microphone height $z$ above the speaker face.
The frequencies $f$ and heights $z$
are motivated by the parameters of the heterodyne interferometer
described in Chapter~\ref{ch:Implementation}.
Specifically, the interferometer spatial bandwidth
$|k| \leq \SI{5}{\per\centi\meter}$
from (\ref{eq:Implementation:kfsv_interferometer_design}) motivates
sound-wave measurements at frequencies
$f \lesssim \SI{30}{\kilo\hertz}$
(i.e.\ $|k| \lesssim \SI{5}{\per\centi\meter}$).
Further, to produce a robust interference signal
during sound-wave calibrations,
the speaker is placed very close
to the edge of the collimated probe beam,
which has 1/e $E$ radius $w_0 = \SI{3.4}{\centi\meter}$;
thus, sound-wave measurements are made at heights
spanning the probe-beam profile
$z
=
\{\SI{2.5}{\centi\meter}, \SI{5.5}{\centi\meter}, \SI{8.5}{\centi\meter}\}$.
The on-axis amplitude of the sound waves as a function of
wavenumber $k$ and height $z$ above the speaker face is shown in
Fig.~\ref{fig:SoundWaveCharacterization:tymphany_on_axis_amplitude}.
As expected, the on-axis amplitude decreases with
increasing distance $z$ from the speaker face.
Further, the on-axis amplitude has a complicated wavenumber dependence, but
it is relatively flat for
$\SI{1}{\per\centi\meter} \lesssim k \lesssim \SI{3.5}{\per\centi\meter}$.

\begin{figure}
  \centering
  \includegraphics[width = \textwidth]{%
    Appendices/SoundWaveCharacterization/figs/tymphany_on_axis_amplitude.pdf}
  \caption[On-axis amplitude of sound waves]{%
    On-axis amplitude of sound waves as a function of
    wavenumber $k$ and height $z$ above the speaker face.
    Note that the amplitude is specified
    as a \emph{peak-to-peak} value.
  }
\label{fig:SoundWaveCharacterization:tymphany_on_axis_amplitude}
\end{figure}


\subsection{Wavefront phasing}
\label{sec:SoundWaveCharacterization:Measurements:phasing}
Characterizing the sound-wave phasing is somewhat more involved
than characterizing the on-axis amplitude,
as it requires measurements at several radial positions $\rho$
for each frequency $f$ and microphone height $z$.
For this reason, the wavefront-phasing measurements
are more coarsely sampled in frequency $f$
than the on-axis amplitude measurements in
Section~\ref{sec:SoundWaveCharacterization:Measurements:amlitude}.
For a given frequency $f$ and height $z$,
the sound-wave phasing is measured by
tracking a point of constant phase in the microphone waveform
as the radial position $\rho$ is varied;
such tracking can be easily accomplished
by triggering the oscilloscope
with a copy of the waveform that is driving the speaker.
To begin the radial scan,
the microphone height $z$ is selected, and
the microphone is displaced from the speaker's symmetry axis
by a few centimeters.
Then, in $\SI{1}{\centi\meter}$ increments,
the microphone is moved radially inwards towards the center;
upon passing through the center,
the radial scan is continued in $\SI{1}{\centi\meter}$ increments
until the sound-wave amplitude becomes negligible.
Note that beginning the radial scan
with a small displacement from the symmetry axis
allows empirical identification of the symmetry-axis location
(by e.g.\ fitting the measured amplitude and/or phasing
and identifying the extremum that occurs at the symmetry axis).

\begin{figure}
  \centering
  \includegraphics[width = \textwidth]{%
    Appendices/SoundWaveCharacterization/figs/tymphany_wavefront_phasing.pdf}
  \caption[Wavefront phasing of sound waves]{%
    Wavefront phasing of sound waves.
    The symbols show the measured time delay $\tau$
    between the wavefront at height $z$ and radial displacement $\rho$ and
    the corresponding on-axis wavefront (i.e.\ same $z$ and $\rho = 0$),
    with each symbol shape corresponding to particular frequency.
    The traces correspond to the time delay
    (\ref{eq:SoundWaveCharacterization:time_delay})
    predicted for spherical waves.
    The close proximity of the measured points to the spherical-wave traces
    indicates that, to lowest order, the waves are approximately spherical
    over the spatial domain and frequencies probed.
  }
\label{fig:SoundWaveCharacterization:tymphany_wavefront_phasing}
\end{figure}

At sufficiently large distances,
the speaker will behave like a point source,
producing sound waves with spherical wavefronts.
This point-source approximation is taken as a reasonable
starting point for the investigation of the wavefront phasing.
If a sound wave is measured on axis at height $z$ above the speaker,
the corresponding wavefront will subsequently arrive
at position $r = (z^2 + \rho^2)$
delayed by a time $\tau$
\begin{equation}
  \tau = \frac{r - z}{c_s},
  \label{eq:SoundWaveCharacterization:time_delay}
\end{equation}
\graffito{\textcolor{red}{value for $c_s$}}
where $c_s$ is the sound speed.
Fig.~\ref{fig:SoundWaveCharacterization:tymphany_wavefront_phasing}
compares the measured time delay to
the time delay predicted for spherical waves
(\ref{eq:SoundWaveCharacterization:time_delay})
as a function of height $z$, radial position $\rho$, and frequency $f$.
Clearly, to lowest order, the waves are approximately spherical
over the spatial domain and frequencies probed.


\subsection{Spatial envelope}
\label{sec:SoundWaveCharacterization:Measurements:envelope}


\section{Sound-wave model}
\label{sec:SoundWaveCharacterization:Model}


\section{Perturbed index of refraction}
\label{sec:SoundWaveCharacterization:PerturbedIndexOfRefractiion}


\bibliographystyle{plainurl}
\bibliography{references}
