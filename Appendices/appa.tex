\chapter{Tables}

\begin{table}[ht]
\label{table:PCI_interferometer}
  \centering
  \renewcommand{\arraystretch}{1.5}% Spread rows out...
  \begin{tabular}{%
    >{\centering}m{3.0cm} >{\centering}m{4.5cm} >{\centering}m{4.5cm}
  }
    \toprule%
    \textbf{Parameter} & \textbf{PCI} & \textbf{Interferometer}
    \tabularnewline%
    \midrule
    \textbf{probe beam} & single CO$_2$ beam & single CO$_2$ beam
    \tabularnewline%
    \textbf{frequency bandwidth}
    & \SI{10}{\kilo\hertz} $ < f < $ \SI{2}{\mega\hertz}
    & \SI{10}{\kilo\hertz} $ < f < $ \SI{2}{\mega\hertz}
    \tabularnewline%
    \textbf{spatial bandwidth}
    & \SI{1.5}{\centi\meter}\ts{-1} $ < k < $ \SI{20}{\centi\meter}\ts{-1}
    & \SI{0}{\centi\meter}\ts{-1} $ < k < $ \SI{5}{\centi\meter}\ts{-1}
    \tabularnewline%
    \toprule%
  \end{tabular}
  \caption{PCI and interferometry have compatible probe beams, comparable
    frequency bandwidths, and \emph{complementary} spatial bandwidths.
    All parameters are for DIII-D's currently implemented PCI--interferometer
    system.
  }%
\end{table}

\clearpage
\newpage
