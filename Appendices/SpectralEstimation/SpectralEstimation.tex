\chapter{Spectral estimation}
\label{app:SpectralEstimation}
In contrast to deterministic processes,
random processes cannot be modeled via an explicit mathematical relationship.
Rather, random processes are characterized
in terms of probabilities and statistical properties.
Any given observation of a random process represents
only one of many possible observations;
each such observation is referred to as
a ``sample'' or a ``realization'' of the random process
and is denoted as $x_k(t)$.
The random process itself consists of
the ensemble of all of the potential observations
and is denoted as $\{x_k(t)\}$.
Random processes can be stationary or nonstationary.
The statistical properties of a stationary random process
do not vary in time, and
the spectral tools discussed below
are all developed for analysis of stationary random processes.


\section{Non-parametric techniques}
\label{app:SpectralEstimation:NonParametric}
Most of the discussion below is distilled from
the seminal work by Bendat and Piersol~\cite{bendat_and_piersol}, and
inquisitive readers are directed there
for a more extensive treatment of the subject.

The windowed, finite Fourier transform $X_k(f, T)$
of a continuous signal $x_k(t)$
sampled for $-T / 2 \leq t < T / 2$
is defined as
\begin{equation}
  X_k(f, T)
  =
  \int_{-T / 2}^{T / 2}
  dt \, [w(t) \cdot x_k(t)] e^{-i \, 2 \pi f t},
  \label{eq:SpectralEstimation:finite_Fourier_transform}
\end{equation}
where $w(t)$ is an arbitrary windowing function.
Typically, the selected windowing function smoothly tapers
as $|t| \rightarrow T / 2$
to minimize side-lobe ``leakage''
that results from discontinuities at the start and end of the sample record.
Further, to prevent power loss, the windowing function
is also typically normalized such that
\begin{equation}
  \frac{1}{T} \int_{-T/2}^{T/2} dt \, [w(t)]^2 = 1.
\end{equation}
The normalized Hanning window is perhaps
the most commonly used windowing function, and
it is used uniformly throughout this work.

For real-valued, stationary random processes $\{x_k(t)\}$ and $\{y_k(t)\}$,
the one-sided \emph{cross-spectral density} function $G_{xy}(f)$ is defined as
\begin{equation}
  G_{xy}(f)
  \equiv
  \lim_{T \rightarrow \infty}
  \frac{2}{T} E \left[ X_k^*(f, T) Y_k(f, T) \right]
  \label{eq:SpectralEstimation:cross_spectral_density_defn}
\end{equation}
for $0 < f < \infty$;
$G_{xy}(f)$ is not defined for $f < 0$, and
it is reduced by a factor of two relative to
(\ref{eq:SpectralEstimation:cross_spectral_density_defn}) at $f = 0$
(the value of $G_{xy}(0)$ is of little relevance to this work).
Note that $E[\cdot]$ is the expectation value operator;
this operator averages over all of the realizations in the ensemble, and
its application ensures that
(\ref{eq:SpectralEstimation:cross_spectral_density_defn})
is a statistically consistent definition of the cross-spectral density
(that is, ensemble averaging is needed for $G_{xy}(f)$
to approach the true cross-spectral density
as $T \rightarrow \infty$). % see pgs. 127, 128 of Bendat & Piersol, 4th ed.
If, in addition to being stationary,
the random process is also \emph{ergodic},
the ensemble average can be replaced
with a time average of $X_k(f, T)$
over successive time slices.
If desired, these time slices may partially overlap.
Unless otherwise noted,
all of the ensemble averages in this work are computed
using this assumption of ergodicity, and
successive slices are selected to overlap by 50\%.

In general $G_{xy}(f)$ is a complex-valued function.
This can be made explicit by writing
\begin{equation}
  G_{xy}(f) = \left| G_{xy}(f) \right| e^{i \alpha_{xy}(f)},
  \label{eq:SpectralEstimation:cross_spectral_density_explicit_complex}
\end{equation}
where $\alpha_{xy}(f)$ is the \emph{phase angle}.
Note that if
$\lim_{T \rightarrow \infty} E[X_k(f, T)] \propto e^{i \alpha_x}$ and
$\lim_{T \rightarrow \infty} E[Y_k(f, T)] \propto e^{i \alpha_y}$, then
\begin{equation}
  \alpha_{xy} = \alpha_y - \alpha_x.
\end{equation}
Further, note that for the special case $\{x_k(t)\} = \{y_k(t)\}$,
$G_{xx}(f)$ is real-valued (i.e.\ $G_{xx}(f) = |G_{xx}(f)|$) and
is referred to as the one-sided \emph{autospectral density} function.

The degree of correlation between random processes
$\{x_k(t)\}$ and $\{y_k(t)\}$ can be easily quantified
with the corresponding spectral density functions.
In particular, the \emph{magnitude-squared coherence} function
$\gamma_{xy}^2(f)$ is defined as
\begin{equation}
  \gamma_{xy}^2(f)
  \equiv
  \frac{|G_{xy}(f)|^2}{G_{xx}(f) G_{yy}(f)},
  \label{eq:SpectralEstimation:magnitude_squared_coherence_defn}
\end{equation}
and it satisfies
\begin{equation}
  0 \leq \gamma_{xy}^2(f) \leq 1
  \label{eq:SpectralEstimation:magnitude_squared_coherence_bounds}
\end{equation}
for $0 \leq f < \infty$.
If $\gamma_{xy}^2(f) = 1$,
$\{x_k(t)\}$ and $\{y_k(t)\}$ are $100\%$ correlated at frequency $f$, and
if $\gamma_{xy}^2(f) = 0$,
$\{x_k(t)\}$ and $\{y_k(t)\}$ are completely uncorrelated at frequency $f$.
Note that the ensemble-averaging operation in
(\ref{eq:SpectralEstimation:cross_spectral_density_defn})
is paramount to the computation
of \emph{informative} values for $\gamma_{xy}^2(f)$;
that is, if ensemble averaging is ignored, and
only single realizations of the random processes are used,
$\gamma_{xy}^2(f) \equiv 1$ for all $f$,
\emph{regardless} of the actual degree of coherence
between between $\{x_k(t)\}$ and $\{y_k(t)\}$.

Care should be taken when computing spectral density estimates.
Table~\ref{table:ToroidalCorrelation:spectral_estimate_random_errors}
summarizes the random errors associated with the estimates
of various spectral properties.
Note that the number of realizations $N_r$ used
in the computation of the ensemble average
is a parameter that can be specified
at the time of analysis and that
increasing $N_r$ reduces the random errors of each spectral estimate.
(While increased $\gamma_{xy}^2(f)$ also reduces random errors,
$\gamma_{xy}^2(f)$ is an intrinsic property of the data
rather than a parameter that can be specified at the time of analysis).
Further, in various programming languages,
it is not uncommon to ``detrend'' realizations $x_k(t)$ and $y_k(t)$
by subtracting the signal mean or linear trend
prior to application of
(\ref{eq:SpectralEstimation:cross_spectral_density_defn}).
However, the author anecdotally notes that his experience with
such detrending can lead to values of $\gamma_{xy}^2(f)$
that unphysically exceed the bounds established in
(\ref{eq:SpectralEstimation:magnitude_squared_coherence_bounds}).
The subtleties of such discrepancies were not investigated further, and
no such detrending was applied to signals
prior to spectral computations in this work;
instead, if needed, signals are high-pass filtered as described in
Section~\ref{sec:Implementation:DataPreparation:high_pass_filtering}.

\begin{table}[t]
  \centering
  \renewcommand{\arraystretch}{1.5}% Spread rows out...
  \begin{tabular}{%
    >{\centering}m{5.0cm} >{\centering}m{5.0cm}
  }
    \toprule%
    \textbf{Spectral estimate}
    & \textbf{Random error} \cite{bendat_and_piersol}
    \tabularnewline%
    \midrule
    $G_{xy}(f)$
    & $\varepsilon \left[G_{xy}(f) \right]
    =
    \frac{1}{|\gamma_{xy}(f)| \sqrt{N_r}}$
    \tabularnewline%
    $\alpha_{xy}(f)$
    & s.d.$\left[ \alpha_{xy}(f) \right]
    \approx
    \frac{[1 - \gamma_{xy}^2(f)]^{1/2}}{|\gamma_{xy}(f)| \sqrt{2 N_r}}$
    \tabularnewline%
    $\gamma_{xy}^2(f)$
    & $\varepsilon \left[ \gamma_{xy}^2(f) \right]
    =
    \frac{\sqrt{2} [1 - \gamma_{xy}^2(f)]}{|\gamma_{xy}(f)| \sqrt{N_r}}$
    \tabularnewline%
    \toprule%
  \end{tabular}
  \caption[Random errors in spectral estimates]{%
    Random errors in estimates of spectral properties are functions of
    the number of realizations $N_r$ used
    in the computation of the ensemble average and
    the coherence magnitude $|\gamma_{xy}(f)|$.
    Here, s.d$[\cdot]$ represents the standard deviation of the estimate, and
    $\varepsilon[\cdot]$ represents the standard deviation of the estimate
    \emph{normalized} to the true value of the spectral property.
    }%
\label{table:ToroidalCorrelation:spectral_estimate_random_errors}
\end{table}


\section{Parametric techniques}
\label{app:SpectralEstimation:Parametric}


\bibliographystyle{plainurl}
\bibliography{references}
