\chapter{Some identities for the PCI wavenumber response}
\label{app:PCIResponseIdentities}


\section{$\mathcal{E}(r_m, k)$ in the beam's near field}
The effect of wavenumber-dependent manipulation $T(k_x)$
on the $m$\ts{th} scattered beam is given by
the complex-valued function $\mathcal{E}(\vect{r}_m, k)$ as defined in
(\ref{eq:InterferometricMethods:mth_diffracted_beam_kx_filtered_transformation}),
which is repeated here for completeness
\begin{equation}
  \begin{aligned}
    \mathcal{E}(\vect{r}_m, k)
    &=
    \frac{e^{-i m k x_m}}{2 \pi}
    \\
    &\quad \times
    \int dx' \,
    \exp\left[ \frac{-x'^2}{w(z_m)^2} \right]
    \exp\left\{%
      i \left[%
        m k x'
        +
        \frac{k_0 x'^2}{2 R(z_m)}
      \right]
    \right\}
    \\
    &\quad \times
    \int dk_x \,
    T(k_x)
    e^{i k_x (x_m - x')}
  \end{aligned}
  \label{eq:PCIResponseIdentities:mth_diffracted_beam_kx_filtered_transformation_full}
\end{equation}
The integrals are over the full domain of $x'$ and $k_x$, but
contributions to the integral
from regions outside of $|x'| \lesssim w(z)$
are suppressed by the Gaussian envelope such that
\begin{equation}
  \frac{k_0 x'^2}{2 R(z_m)}
  \lesssim
  \frac{k_0 [w(z)]^2}{2 R(z)},
  \label{eq:PCIResponseIdentities:curvature_transformation_constraint_general}
\end{equation}
where the approximations
$w(z_m) \approx w(z)$ and $R(z_m) \approx R(z)$
have been used.
Now, in the beam's near field
($z \ll z_R$, which is often experimentally relevant),
the beam's waist and radius of curvature are
\begin{align}
  w(z) &\approx w_0,
  \\
  R(z) &\approx \frac{z_R^2}{z},
\end{align}
as can be easily verified by examining the definitions in
(\ref{eq:InterferometricMethods:Gaussian_beam_width}) and
(\ref{eq:InterferometricMethods:Gaussian_beam_radius_of_curvature}),
respectively.
Thus,
(\ref{eq:PCIResponseIdentities:curvature_transformation_constraint_general})
becomes
\begin{align}
  \frac{k_0 x'^2}{2 R(z_m)}
  &\lesssim
  \frac{k_0 [w(z)]^2}{2 R(z)}
  \notag \\
  &\approx
  \frac{k_0 w_0^2}{2 (z_R^2 / z)}
  \notag \\
  &= \frac{z}{z_R}
  \notag \\
  &\ll 1,
  \label{eq:PCIResponseIdentities:curvature_transformation_constraint_near_field}
\end{align}
where $z / z_R \ll 1$ follows from the near-field assumption.
Thus, in the beam's near field,
the curvature-induced phase factor is negligible, and
$\mathcal{E}(\vect{r}_m, k)$ reduces to
\begin{equation}
  \begin{aligned}
    \mathcal{E}(\vect{r}_m, k)
    &\approx
    \frac{e^{-i m k x_m}}{2 \pi}
    \int dx' \,
    e^{-\left[ x' / w(z_m) \right]^2}
    e^{i m k x'}
    \\
    &\qquad \times
    \int dk_x \,
    T(k_x)
    e^{i k_x (x_m - x')}.
  \end{aligned}
  \label{eq:PCIResponseIdentities:mth_diffracted_beam_kx_filtered_transformation_near_field}
\end{equation}
This near-field assumption will be implicit
in the remainder of the discussion about PCI.


\section{PCI's $\mathcal{E}(r_m, k)$}
The transfer function of the PCI phase plate can be described as
\begin{equation}
  \begin{aligned}
    T(k_x)
    &=
    i \sqrt{\eta} \, H(k_g - |k_x|)
    \\
    &\quad +
    H(|k_x| - k_g)
    H(k_D - |k_x|),
  \end{aligned}
  \label{eq:PCIResponseIdentities:phase_plate_transfer_function}
\end{equation}
where $H(x)$ is the Heaviside step function defined as
\begin{equation}
  H(x)
  =
  \begin{cases}
    0, \quad &x < 0 \\
    1, \quad &x \geq 0
  \end{cases},
  \label{eq:PCIResponseIdentities:Heaviside_step_function}
\end{equation}
$\eta$ is the reflectivity of the phase-plate groove, and
$k_g$ and $k_D$ are the low-$k$ and high-$k$ cutoffs of the phase plate
as defined in
(\ref{eq:InterferometricMethods:pci_kmin_engineering}) and
(\ref{eq:InterferometricMethods:pci_kmax_engineering}), respectively.
Note that the first term on the right-hand side of
(\ref{eq:PCIResponseIdentities:phase_plate_transfer_function})
corresponds to reflection from the phase-plate groove, while
the second term corresponds to reflection
from the non-grooved portion of the phase plate (i.e.\ the ``face'').
Thus, for PCI
\begin{equation}
  \begin{aligned}
    \mathcal{E}(\vect{r}_m, k)
    &=
    \frac{e^{-i m k x_m}}{2 \pi}
    \int dx' \,
    e^{-\left[ x' / w(z_m) \right]^2}
    e^{i m k x'}
    \\
    &\begin{aligned}
      \quad
      \times
      \Biggl\{%
        &\int_{-k_D}^{-k_g} dk_x' \,
        e^{i k_x' (x_m - x')}
        \\
        &+
        i \sqrt{\eta}
        \int_{-k_g}^{k_g} dk_x' \,
        e^{i k_x' (x_m - x')}
        \\
        &+
        \int_{k_g}^{k_D} dk_x' \,
        e^{i k_x' (x_m - x')}
      \Biggr\}.
    \end{aligned}
  \end{aligned}
  \label{eq:PCIResponseIdentities:mth_diffracted_beam_kx_filtered_transformation_near_field_integrals}
\end{equation}


\section{Some useful integrals for evaluation of $\mathcal{E}(r_m, k)$}


\subsection{Finite-domain inverse Fourier transforms of unity}
Note that
\begin{equation}
  \int_{k_1}^{k_2} dk_x
  e^{i k_x x}
  =
  \frac{e^{i k_2 x} - e^{i k_1 x}}{ix}.
\end{equation}
Now, if $k_1 = -k_2$, this simplifies to
\begin{equation}
  \int_{-k_2}^{k_2} dk_x
  e^{i k_x x}
  =
  2 k_2 \sinc \left( \frac{k_2 x}{\pi} \right),
  \label{eq:PCIResponseIdentities:finite_domain_inverse_FT_groove}
\end{equation}
where
\begin{equation}
  \sinc(x) = \frac{\sin(\pi x)}{\pi x}
  \label{eq:PCIResponseIdentities:normalized_sinc}
\end{equation}
is the normalized sinc function;
note that sinc is an \emph{even} function.
Finally, note that
\begin{align}
  \int_{-k_2}^{-k_1}
  &dk_x
  e^{i k_x x}
  +
  \int_{k_1}^{k_2} dk_x
  e^{i k_x x}
  \notag \\
  &=
  \int_{-k_2}^{k_2} dk_x
  e^{i k_x x}
  -
  \int_{-k_1}^{k_1} dk_x
  e^{i k_x x}
  \notag \\
  &=
  2 k_2 \sinc \left( \frac{k_2 x}{\pi} \right)
  -
  2 k_1 \sinc \left( \frac{k_1 x}{\pi} \right).
  \label{eq:PCIResponseIdentities:finite_domain_inverse_FT_face}
\end{align}
Using (\ref{eq:PCIResponseIdentities:finite_domain_inverse_FT_groove}) and
(\ref{eq:PCIResponseIdentities:finite_domain_inverse_FT_face}),
it is easy to see that
(\ref{eq:PCIResponseIdentities:mth_diffracted_beam_kx_filtered_transformation_near_field_integrals})
becomes
\begin{equation}
  \begin{aligned}
    \mathcal{E}(\vect{r}_m, k)
    &=
    \frac{e^{-i m k x_m}}{\pi}
    \int dx' \,
    e^{-\left[ x' / w(z) \right]^2}
    e^{i m k x'}
    \\
    &\begin{aligned}
      \quad
      \times
      \Biggl\{%
        &k_D \sinc\left[ \frac{k_D}{\pi} (x' - x_m) \right]
        \\
        &-
        k_g \sinc\left[ \frac{k_g}{\pi} (x' - x_m) \right]
        \\
        &+
        i \sqrt{\eta}
        k_g \sinc\left[ \frac{k_g}{\pi} (x' - x_m) \right]
      \Biggr\}.
    \end{aligned}
  \end{aligned}
  \label{eq:PCIResponseIdentities:mth_diffracted_beam_kx_filtered_transformation_near_field_sincs}
\end{equation}


\subsection{Integral of offset sinc with complex-Gaussian weighting}
Note that
(\ref{eq:PCIResponseIdentities:mth_diffracted_beam_kx_filtered_transformation_near_field_sincs})
consists of several integrals of the form
\begin{equation}
  I
  \equiv
  \frac{b}{\pi}
  \int dx \,
  e^{-a x^2}
  e^{i c x}
  \sinc\left[ \frac{b}{\pi} (x - x_0) \right],
  \label{eq:PCIResponseIdentities:I_x0_abc_definition}
\end{equation}
where $a > 0$ and $x_0$, $b$, and $c$ are real.
While daunting, the integral can be evaluated ``analytically''
in terms of complex error functions as follows
\begin{align}
  I
  &=
  \frac{b}{\pi}
  \int dx \,
  e^{-a x^2}
  e^{i c x}
  \sinc\left[ \frac{b}{\pi} (x - x_0) \right]
  \notag \\
  &=
  \frac{1}{\pi}
  \int dx \,
  e^{-a x^2}
  e^{i c x}
  \cdot
  \frac{\sin[b (x - x_0)]}{x - x_0}
  \notag \\
  &=
  \frac{1}{\pi}
  \int_{0}^{b} d\beta
  \int dx \,
  e^{-a x^2}
  e^{i c x}
  \cos[\beta (x - x_0)]
  \notag \\
  &=
  \frac{1}{2 \pi}
  \int_{0}^{b} d\beta
  \int dx \,
  e^{-a x^2}
  e^{i c x}
  \left[%
    e^{i \beta (x - x_0)}
    +
    e^{-i \beta (x - x_0)}
  \right]
  \notag \\
  &\begin{aligned}
    &=
    \frac{1}{2 \pi}
    \int_{0}^{b} d\beta
    \int dx \,
    e^{-a x^2}
    e^{i [(\beta + c) x - \beta x_0]}
    \\
    &\quad+
    \frac{1}{2 \pi}
    \int_{0}^{b} d\beta
    \int dx \,
    e^{-a x^2}
    e^{i [-(\beta - c) x + \beta x_0]}
  \end{aligned}
  \notag \\
  &\begin{aligned}
    &=
    \frac{1}{2 \pi}
    \int_{0}^{b} d\beta
    \int dx \,
    e^{-a x^2}
    e^{i [(\beta + c) x - \beta x_0]}
    \\
    &\quad-
    \frac{1}{2 \pi}
    \int_{0}^{-b} d\beta'
    \int dx \,
    e^{-a x^2}
    e^{i [(\beta' + c) x - \beta' x_0]}
  \end{aligned}
  \notag \\
  &=
  \frac{1}{2 \pi}
  \int_{-b}^{b} d\beta
  \int dx \,
  e^{-a x^2}
  e^{i [(\beta + c) x - \beta x_0]}
  \notag \\
  &=
  \frac{1}{2 \pi}
  \int_{-b}^{b} d\beta \,
  e^{-(\beta + c)^2 / 4 a}
  e^{-i \beta x_0}
  \int dx \,
  e^{-a [x - i (\beta + c) / 2a]^2}
  \notag \\
  &=
  \frac{1}{2 \sqrt{\pi a}}
  \int_{-b}^{b} d\beta \,
  e^{-(\beta + c)^2 / 4 a}
  e^{-i \beta x_0}
  \notag \\
  &=
  \frac{1}{2 \sqrt{\pi a}}
  e^{i c x_0}
  \int_{c - b}^{c + b} d\beta' \,
  e^{-\beta'^2 / 4 a}
  e^{-i \beta' x_0}
  \notag \\
  &=
  \frac{1}{2 \sqrt{\pi a}}
  e^{-a x_0^2}
  e^{i c x_0}
  \int_{c - b}^{c + b} d\beta' \,
  e^{-(\beta' + i 2 a x_0)^2 / 4 a}
  \notag \\
  &=
  \frac{1}{\sqrt{\pi}}
  e^{-a x_0^2}
  e^{i c x_0}
  \int_{u(c - b, a, x_0)}^{u(c + b, a, x_0)} du \,
  e^{-u^2}
  \notag \\
  &\begin{aligned}
    =
    \frac{1}{2}
    e^{-a x_0^2}
    e^{i c x_0}
    \bigl\{
      &\erf[u(c + b, a, x_0)]
      \\
      &-
      \erf[u(c - b, a, x_0)]
    \bigr\},
  \end{aligned}
  \label{eq:PCIResponseIdentities:I_x0_abc_evaluated}
\end{align}
where the error function is defined for complex argument $z$ as
\begin{equation}
  \erf(z)
  =
  \frac{2}{\sqrt{\pi}}
  \int_0^z e^{-t^2} dt,
  \label{eq:PCIResponseIdentities:error_function}
\end{equation}
and
\begin{equation}
  u(\gamma, a, x_0) = \frac{1}{2 \sqrt{a}} (\gamma + i 2 a x_0).
\end{equation}

Now, substituting the appropriate values
into (\ref{eq:PCIResponseIdentities:I_x0_abc_evaluated})
\begin{equation}
  x_0 \equiv x_m,
  \qquad
  a \equiv \frac{1}{w(z)^2},
  \qquad
  b \equiv k_j,
  \qquad
  c \equiv mk
  \notag
\end{equation}
yields
\begin{equation}
  I
  =
  \frac{1}{2}
  e^{-[x_m / w(z_m)]^2}
  e^{i m k x_m}
  \mathcal{D}(\vect{r}_m, k, k_j),
  \label{eq:PCIResponseIdentities:I_x0_abc_evaluated_lab_parameters}
\end{equation}
where the difference function $\mathcal{D}$ is defined as
\begin{equation}
  \mathcal{D}(\vect{r}_m, k, k_j)
  =
  \erf[u(\vect{r}_m, k, k_j)]
  -
  \erf[u(\vect{r}_m, k, -k_j)],
  \label{eq:PCIResponseIdentities:difference_function}
\end{equation}
and
\begin{equation}
  u(\vect{r}_m, k, k_j)
  =
  \frac{w(z_m)}{2}
  \left[%
    (m k + k_j)
    +
    i \frac{2 x_m}{w(z_m)^2}
  \right].
  \label{eq:PCIResponseIdentities:u}
\end{equation}
With these definitions
(\ref{eq:PCIResponseIdentities:mth_diffracted_beam_kx_filtered_transformation_near_field_sincs})
readily reduces to
\begin{equation}
  \begin{aligned}
    \mathcal{E}(\vect{r}_m, k)
    &=
    \frac{1}{2}
    e^{-[x_m / w(z_m)]^2}
    \\
    &\quad\times
    \left[%
      \mathcal{D}(\vect{r}_m, k, k_D)
      +
      (i \sqrt{\eta} - 1)
      \mathcal{D}(\vect{r}_m, k, k_g)
    \right].
  \end{aligned}
  \label{eq:PCIResponseIdentities:mth_diffracted_beam_kx_filtered_transformation_near_field_difference_functions}
\end{equation}


\section{Symmetries and degeneracies in the image plane}
The plasma midplane sits at the object plane
of a magnification-$M$ imaging system.
The probe beam's waist also nominally sits at this object plane,
with a 1/e $E$ radius of $w_{0,\object}$, and
the plasma dimensions are far smaller than the corresponding Rayleigh length
such that throughout the plasma volume
\begin{equation}
  w(z) \approx w_{0,\object}.
\end{equation}
As discussed in
Appendix~\ref{app:ImagingSystems},
the waist of the imaged beam
does \emph{not} necessarily occur at the image plane.
However, the derivation of
(\ref{eq:PCIResponseIdentities:mth_diffracted_beam_kx_filtered_transformation_near_field})
required assuming that the beam was well within its Rayleigh range
(i.e.\ $|z_{\image}| \ll z_{R,\image}$).
Thus, valid application of any of the above derived results to the image plane
requires that the image plane and the waist of the imaged beam
approximately overlap such that
\begin{equation}
  w(z_{\image}) \approx w_{0,\image} \approx M w_{0,\object},
  \label{eq:PCIResponseIdentities:criterion_on_imaged_beam_radius}
\end{equation}
where $w_{0,\image}$ is the 1/e $E$ radius of the imaged beam at its waist.

The coordinate system and properties of the $m$\ts{th} scattered beam
also have certain symmetries in the image plane.
In particular, according to the image-plane coordinate transformation in
(\ref{eq:InterferometricMethods:coordinate_transformation_imaging_plane}),
\begin{align}
  x_{-m,\image}
  &=
  x_{\image} \cos \left( \frac{\theta_{-m}}{M} \right)
  \notag \\
  &=
  x_{\image} \cos \left( \frac{-\theta_m}{M} \right)
  \notag \\
  &=
  x_{\image} \cos \left( \frac{\theta_m}{M} \right)
  \notag \\
  &=
  x_{m,\image}.
  \label{eq:PCIResponseIdentities:x_symmetry}
\end{align}
Additionally, again from the image-plane coordinate transformation in
(\ref{eq:InterferometricMethods:coordinate_transformation_imaging_plane}),
$z_{-m,\image} = z_{\image} + x_{\image} \sin (\theta_{-m} / M)$ such that
\begin{equation}
  w(z_{-m,\image}) \approx w(z_{\image}) \approx w(z_{m,\image})
  \label{eq:PCIResponseIdentities:w_symmetry}
\end{equation}
are very good approximations when $\theta_m / M \ll 1$, as is typical.
The above symmetries lead to several other symmetries and degeneracies
that will be useful for evaluating
$\mathcal{E}(\vect{r}_m, k)$ in the image plane.


\subsection{Properties of $u$ in the image plane}
The complex-valued function $u$ is defined in
(\ref{eq:PCIResponseIdentities:u}).
Referencing symmetry properties
(\ref{eq:PCIResponseIdentities:x_symmetry}) and
(\ref{eq:PCIResponseIdentities:w_symmetry}) and
cleverly rearranging terms,
it readily follows that $u$ is anti-Hermitian
with respect to $m$ and $k_{j,\image}$:
\begin{align}
  u(\vect{r}_{-m, \image}, k_{\image}, -k_{j,\image})
  &=
  \frac{w(z_{-m,\image})}{2}
  \left[%
    (-m k_{\image} - k_{j,\image})
    +
    i \frac{2 x_{-m,\image}}{w(z_{-m,\image})^2}
  \right]
  \notag \\
  &\approx
  \frac{w(z_{m,\image})}{2}
  \left[%
    (-m k_{\image} - k_{j,\image})
    +
    i \frac{2 x_{m,\image}}{w(z_{m,\image})^2}
  \right]
  \notag \\
  &=
  \frac{w(z_{m,\image})}{2}
  \left[%
    -(m k_{\image} + k_{j,\image})
    +
    i \frac{2 x_{m,\image}}{w(z_{m,\image})^2}
  \right]
  \notag \\
  &=
  \frac{-w(z_{m,\image})}{2}
  \left[%
    (m k_{\image} + k_{j,\image})
    -
    i \frac{2 x_{m,\image}}{w(z_{m,\image})^2}
  \right]
  \notag \\
  &=
  \frac{-w(z_{m,\image})}{2}
  \left[%
    (m k_{\image} + k_{j,\image})
    +
    i \frac{2 x_{m,\image}}{w(z_{m,\image})^2}
  \right]^*
  \notag \\
  &=
  -[u(\vect{r}_{m, \image}, k_{\image}, k_{j,\image})]^*,
  \label{eq:PCIResponseIdentities:u_symmetry}
\end{align}
where $z^*$ indicates the complex conjugate of $z$.
Further, referencing
(\ref{eq:PCIResponseIdentities:criterion_on_imaged_beam_radius}),
note that
\begin{align}
  u(\vect{r}_{m, \image}, k_{\image}, k_{j,\image})
  &=
  \frac{w(z_{\image})}{2}
  \left[%
    (m k_{\image} + k_{j,\image})
    +
    i \frac{2 x_{\image}}{w(z_{\image})^2}
  \right]
  \notag \\
  &\approx
  \frac{M w_{0,\object}}{2}
  \left[%
    \left(\frac{m k}{M} + \frac{k_j}{M}\right)
    +
    i \frac{2 M x_{\object}}{(M w_{0,\object})^2}
  \right]
  \notag \\
  &=
  \frac{w_{0,\object}}{2}
  \left[%
    \left(m k + k_j \right)
    +
    i \frac{2 x_{\object}}{(w_{0,\object})^2}
  \right]
  \notag \\
  &=
  u(\vect{r}_{m,\object}, k, k_{j,\object});
\end{align}
that is,
$u(\vect{r}_{m, \image}, k_{\image},  k_{j,\image})$ and
$u(\vect{r}_{m, \object}, k, k_{j,\object})$ are geometrically \emph{similar},
as is expected in an imaging system.


\subsection{Properties of the error function}
The error function has two useful properties
that will be exploited shortly.
First, the error function is \emph{odd}
\begin{equation}
  \erf(-z) = - \erf(z),
  \label{eq:PCIResponseIdentities:error_function_is_odd}
\end{equation}
as is easily determined by inspection.
Second, the error function \emph{commutes} with complex conjugation
\begin{equation}
  \erf(z^*) = [\erf(z)]^*,
  \label{eq:PCIResponseIdentities:error_function_commutativity}
\end{equation}
where $z^*$ is the complex conjugate of $z$.


\subsection{Properties of $\mathcal{D}$ in the image plane}
The complex-valued difference function $\mathcal{D}$ is defined in
(\ref{eq:PCIResponseIdentities:difference_function}).
Using the anti-Hermitian properties of $u$ from
(\ref{eq:PCIResponseIdentities:u_symmetry}) and
the error-function properties
(\ref{eq:PCIResponseIdentities:error_function_is_odd}) and
(\ref{eq:PCIResponseIdentities:error_function_commutativity}),
it is easy to show that $\mathcal{D}$ is Hermitian with respect to $m$:
\begin{align}
  &\begin{aligned}
    \mathcal{D}(\vect{r}_{-m, \image}, k_{\image}, k_{j,\image})
    &=
    \erf[u(\vect{r}_{-m, \image}, k_{\image}, k_{j,\image})]
    \\
    &\quad
    -
    \erf[u(\vect{r}_{-m,\image}, k_{\image}, -k_{j,\image})]
  \end{aligned}
  \notag \\
  &\begin{aligned}
    \phantom{\mathcal{D}(\vect{r}_{-m, \image}, k_{\image}, k_{j,\image})}
    &=
    \erf\{-[u(\vect{r}_{m,\image}, k_{\image}, -k_{j,\image})]^*\}
    \\
    &\quad
    -
    \erf\{-[u(\vect{r}_{m,\image}, k_{\image}, k_{j,\image})]^*\}
  \end{aligned}
  \notag \\
  &\begin{aligned}
    \phantom{\mathcal{D}(\vect{r}_{-m, \image}, k_{\image}, k_{j,\image})}
    &=
    -\{\erf[u(\vect{r}_{m,\image}, k_{\image}, -k_{j,\image})]\}^*
    \\
    &\quad
    +
    \{\erf[u(\vect{r}_{m,\image}, k_{\image}, k_{j,\image})]\}^*
  \end{aligned}
  \notag \\
  &\begin{aligned}
    \phantom{\mathcal{D}(\vect{r}_{-m, \image}, k_{\image}, k_{j,\image})}
    &=
    \{\erf[u(\vect{r}_{m,\image}, k_{\image}, k_{j,\image})]
    \\
    &\quad
    -
    \erf[u(\vect{r}_{m,\image}, k_{\image}, -k_{j,\image})]\}^*
  \end{aligned}
  \notag \\
  &\phantom{\mathcal{D}(\vect{r}_{-m, \image}, k_{\image}, k_{j,\image})}
  =
  \mathcal{D}^*(\vect{r}_{m,\image}, k_{\image}, k_{j,\image}).
  \label{eq:PCIResponseIdentities:difference_function_symmetry}
\end{align}
The above symmetry relation also implies a degeneracy when $m = 0$; namely,
$\mathcal{D}(\vect{r}_{0,\image}, k_{\image}, k_{j,\image})
=
\mathcal{D}^*(\vect{r}_{0,\image}, k_{\image}, k_{j,\image})$,
which proves that
$\mathcal{D}(\vect{r}_{0,\image}, k_{\image}, k_{j,\image})$
is purely \emph{real}.


\subsection{Properties of $\mathcal{E}$ in the image plane}
Eq. (\ref{eq:PCIResponseIdentities:difference_function_symmetry})
allows
(\ref{eq:PCIResponseIdentities:mth_diffracted_beam_kx_filtered_transformation_near_field_difference_functions})
to be rewritten as
\begin{equation}
  \begin{aligned}
    \mathcal{E}(\vect{r}_{m,\image}, k_{\image})
    &=
    e^{-[x_{m,\image} / w(z_{m,\image})]^2}
    \\
    &\quad\times
    \left[%
      F(\vect{r}_{m,\image}, k_{\image})
      +
      G(\vect{r}_{m,\image}, k_{\image})
    \right],
  \end{aligned}
  \label{eq:PCIResponseIdentities:transformation_Hermitian_decomposed}
\end{equation}
where
\begin{align}
  F(\vect{r}_{m,\image}, k_{\image})
  &=
  \frac{1}{2}
  \left[%
    \mathcal{D}(\vect{r}_{m,\image}, k_{\image}, k_{D,\image})
    -
    \mathcal{D}(\vect{r}_{m,\image}, k_{\image}, k_{g,\image})
  \right],
  \label{eq:PCIResponseIdentities:transformation_face}
  \\
  G(\vect{r}_{m,\image}, k_{\image})
  &=
  \frac{i \sqrt{\eta}}{2} \,
  \mathcal{D}(\vect{r}_{m,\image}, k_{\image}, k_{g,\image}).
  \label{eq:PCIResponseIdentities:transformation_groove}
\end{align}
Here, the notation is mnemonic:
the non-grooved portion of the phase plate (i.e.\ the ``face'')
acts on the $m$\ts{th} scattered beam via $F$, while
the phase-plate groove acts on the $m$\ts{th} scattered beam via $G$.
Note that $F$ is Hermitian with respect to $m$
\begin{equation}
  F(\vect{r}_{-m,\image}, k_{\image}) = F^*(\vect{r}_{m,\image}, k_{\image}),
  \label{eq:PCIResponseIdentities:mth_beam_interaction_with_face_hermitian}
\end{equation}
while $G$ is anti-Hermitian with respect to $m$
\begin{equation}
  G(\vect{r}_{-m,\image}, k_{\image}) = -G^*(\vect{r}_{m,\image}, k_{\image}).
  \label{eq:PCIResponseIdentities:mth_beam_interaction_with_groove_antihermitian}
\end{equation}

Note that the above symmetries also imply a degeneracy when $m = 0$.
Specifically,
(\ref{eq:PCIResponseIdentities:mth_beam_interaction_with_face_hermitian})
states that
$F(\vect{r}_{0,\image}, k_{\image}) = F^*(\vect{r}_{0,\image}, k_{\image})$;
that is, $F(\vect{r}_{0,\image}, k_{\image})$ is purely \emph{real}
\begin{equation}
  F(\vect{r}_{0,\image}, k_{\image})
  =
  \real[F(\vect{r}_{0,\image}, k_{\image})].
\end{equation}
Similarly,
(\ref{eq:PCIResponseIdentities:mth_beam_interaction_with_groove_antihermitian})
states that
$G(\vect{r}_{0,\image}, k_{\image}) = -G^*(\vect{r}_{0,\image}, k_{\image})$;
that is, $G(\vect{r}_{0,\image}, k_{\image})$ is purely \emph{imaginary}
\begin{equation}
  G(\vect{r}_{0,\image}, k_{\image})
  =
  i \cdot \imag[G(\vect{r}_{0,\image}, k_{\image})].
\end{equation}

Another set of useful degeneracies occurs
when the fluctuation wavenumber vanishes (i.e.\ $k_{\image} = 0$).
Note that the $k_{\image}$ dependence of $F$ and $G$
only appears as the \emph{product} $(m \cdot k_{\image})$.
Just as $m = 0$ at finite $k_{\image}$ yields $(m \cdot k_{\image}) = 0$,
$k_{\image} = 0$ at finite $m$ also gives $(m \cdot k_{\image}) = 0$;
thus,
\begin{align}
  F(\vect{r}_{m,\image}, k_{\image}=0) &= F(\vect{r}_{0,\image}, k_{\image}),
  \label{eq:PCIResponseIdentities:mth_beam_interaction_with_face_for_wavenumber_0}
  \\
  G(\vect{r}_{m,\image}, k_{\image}=0) &= G(\vect{r}_{0,\image}, k_{\image}),
  \label{eq:PCIResponseIdentities:mth_beam_interaction_with_groove_for_wavenumber_0}
\end{align}
where the approximation $w(z_{m,\image}) \approx w(z_{0,\image})$
has been used.
The PCI amplitude response $A_{\text{pci}}(k_{\image}, x_{\image})$ is given by
(\ref{eq:InterferometricMethods:PCI_amplitude_response});
of particular relevance is that
$A_{\text{pci}}(k_{\image}, x_{\image}) \propto A(k_{\image}, x_{\image})$,
where $A(k_{\image}, x_{\image}) = (A_I^2 + A_Q^2)^{1/2}$ is a real number
and $A_I$ and $A_Q$ are given by
(\ref{eq:InterferometricMethods:PCI_response_AI}) and
(\ref{eq:InterferometricMethods:PCI_response_AQ}), respectively.
Note that the notational shorthand
$F_m \equiv F(\vect{r}_{m,\image}, k_{\image})$ and
$G_m \equiv G(\vect{r}_{m,\image}, k_{\image})$
is being used in the expressions for $A_I$ and $A_Q$.
Using
(\ref{eq:PCIResponseIdentities:mth_beam_interaction_with_face_for_wavenumber_0})
and
(\ref{eq:PCIResponseIdentities:mth_beam_interaction_with_groove_for_wavenumber_0}),
the expression for $A_I$ in the low-$k$ limit becomes
\begin{align}
  \lim_{k_{\image} \rightarrow 0}
  A_I(k_{\image}, x_{\image})
  &=
  \imag(G_0) \left[ \lim_{k_{\image} \rightarrow 0} \real(F_1) \right]
  -
  \real(F_0) \left[ \lim_{k_{\image} \rightarrow 0} \imag(G_1) \right]
  \notag \\
  &=
  \imag(G_0) \real(F_0)
  -
  \real(F_0) \imag(G_0)
  \notag \\
  &=
  0.
  \notag
\end{align}
Similarly, the expression for $A_Q$ in the low-$k$ limit becomes
\begin{align}
  \lim_{k_{\image} \rightarrow 0}
  A_Q(k_{\image}, x_{\image})
  &=
  \imag(G_0) \left[ \lim_{k_{\image} \rightarrow 0} \imag(F_1) \right]
  +
  \real(F_0) \left[ \lim_{k_{\image} \rightarrow 0} \real(G_1) \right]
  \notag \\
  &=
  \imag(G_0) \imag(F_0)
  +
  \real(F_0) \real(G_0)
  \notag \\
  &=
  \imag(G_0) \cdot 0
  +
  \real(F_0) \cdot 0
  \notag \\
  &=
  0,
  \notag
\end{align}
where the third line follows from the fact that
$F_0$ is purely real and $G_0$ is purely imaginary.
As $A_I$ and $A_Q$ vanish at $k = 0$,
$A$ and $A_{\text{pci}}$ must also vanish at $k = 0$.
This is in agreement with expectations:
as $k \rightarrow 0$, the upscattered and downscattered beams
fall into the phase-plate groove,
reducing the phase contrast, and
the response vanishes fully for $k = 0$.
