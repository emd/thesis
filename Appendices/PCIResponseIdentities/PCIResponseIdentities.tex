\chapter{Some identities for the PCI wavenumber response}
\label{app:PCIResponseIdentities}


\section{Phase factor in the beam's near field}
The effect of wavenumber-dependent manipulation $T(k_x)$
on the $m$\ts{th} scattered beam is given by
the phase factor $\mathcal{P}(x, m, k)$ as defined in
(\ref{eq:InterferometricMethods:mth_diffracted_beam_kx_filtered_phase_factor}),
which is repeated here for completeness
\begin{equation}
  \begin{aligned}
    \mathcal{P}(x, m, k)
    &=
    \frac{1}{2 \pi}
    \int dx' \,
    \exp\left[ \frac{-x'^2}{w(z)^2} \right]
    \exp\left\{%
      i \left[%
        m k x'
        +
        \frac{k_0 x'^2}{2 R(z)}
      \right]
    \right\}
    \\
    &\qquad \times
    \int dk_x' \,
    T(k_x')
    e^{i k_x' (x_m - x')}
  \end{aligned}
  \label{eq:PCIResponseIdentities:mth_diffracted_beam_kx_filtered_phase_factor_full}
\end{equation}
The integrals are over the full domain of $x'$ and $k_x'$, but
note that contributions to the integral
from regions outside of $|x'| \lesssim w(z)$
are suppressed by the Gaussian envelope such that
the maximum value of the curvature-induced phase factor obeys
\begin{equation}
  \left[ \frac{k_0 x'^2}{2 R(z)} \right]_{\text{max}}
  \sim
  \frac{k_0 [w(z)]^2}{2 R(z)}
  \label{eq:PCIResponseIdentities:curvature_phase_factor_constraint_general}
\end{equation}
Now, in the beam's near field
($z \ll z_R$, which is often experimentally relevant),
the beam's waist and radius of curvature are
\begin{align}
  w(z) &\approx w_0
  \\
  R(z) &\approx \frac{z_R^2}{z}
\end{align}
as can be easily verified by examining the definitions in
(\ref{eq:InterferometricMethods:Gaussian_beam_width}) and
(\ref{eq:InterferometricMethods:Gaussian_beam_radius_of_curvature}),
respectively.
Thus,
(\ref{eq:PCIResponseIdentities:curvature_phase_factor_constraint_general})
becomes
\begin{align}
  \left[ \frac{k_0 x'^2}{2 R(z)} \right]_{\text{max}}
  &\sim
  \frac{k_0 [w(z)]^2}{2 R(z)}
  \notag \\
  &\approx
  \frac{k_0 w_0^2}{2 (z / z_R)}
  \notag \\
  &= \frac{z}{z_R}
  \notag \\
  &\ll 1
  \label{eq:PCIResponseIdentities:curvature_phase_factor_constraint_near_field}
\end{align}
where $z / z_R \ll 1$ follows from the near-field assumption.
Thus, in the beam's near field,
the curvature-induced phase factor is negligible, and
$\mathcal{P}(x, m, k)$ reduces to
\begin{equation}
  \begin{aligned}
    \mathcal{P}(x, m, k)
    &=
    \frac{1}{2 \pi}
    \int dx' \,
    e^{-\left[ x' / w(z) \right]^2}
    e^{i m k x'}
    \\
    &\qquad \times
    \int dk_x' \,
    T(k_x')
    e^{i k_x' (x_m - x')}
  \end{aligned}
  \label{eq:PCIResponseIdentities:mth_diffracted_beam_kx_filtered_phase_factor_near_field}
\end{equation}
This near-field assumption will be implicit
in the remainder of the discussion about PCI.


\section{The PCI phase factor}
The transfer function of the PCI phase plate can be described as
\begin{equation}
  \begin{aligned}
    T(k_x)
    &=
    i \sqrt{\eta} \, H(k_g - |k_x|)
    \\
    &\quad +
    H(|k_x| - k_g)
    H(k_D - |k_x|)
  \end{aligned}
  \label{eq:PCIResponseIdentities:phase_plate_transfer_function}
\end{equation}
where $H(x)$ is the Heaviside step function defined as
\begin{equation}
  H(x)
  =
  \begin{cases}
    0, \quad &x < 0 \\
    1, \quad &x \geq 0
  \end{cases}
  \label{eq:PCIResponseIdentities:Heaviside_step_function}
\end{equation}
$\eta$ is the reflectivity of the phase-plate groove, and
$k_g$ and $k_D$ are the low-$k$ and high-$k$ cutoffs of the phase plate
as defined in
(\ref{eq:InterferometricMethods:pci_kmin_engineering}) and
(\ref{eq:InterferometricMethods:pci_kmax_engineering}), respectively.
Thus, the PCI phase factor $\mathcal{P}(x, m, k)$ is
\begin{equation}
  \begin{aligned}
    \mathcal{P}(x, m, k)
    &=
    \frac{1}{2 \pi}
    \int dx' \,
    e^{-\left[ x' / w(z) \right]^2}
    e^{i m k x'}
    \\
    &\begin{aligned}
      \quad
      \times
      \Biggl\{%
        &\int_{-k_D}^{-k_g} dk_x' \,
        e^{i k_x' (x_m - x')}
        \\
        &+
        i \sqrt{\eta}
        \int_{-k_g}^{k_g} dk_x' \,
        e^{i k_x' (x_m - x')}
        \\
        &+
        \int_{k_g}^{k_D} dk_x' \,
        e^{i k_x' (x_m - x')}
      \Biggr\}
    \end{aligned}
  \end{aligned}
  \label{eq:PCIResponseIdentities:mth_diffracted_beam_kx_filtered_phase_factor_near_field_integrals}
\end{equation}



\section{The error function}
The error function is defined as
\begin{equation}
  \erf(z)
  =
  \frac{2}{\sqrt{\pi}}
  \int_0^z e^{-t^2} dt
  \label{eq:PCIResponseIdentities:error_function}
\end{equation}
for complex argument $z$.
The error function is \emph{odd}
\begin{equation}
  \erf(-z) = - \erf(z)
  \label{eq:PCIResponseIdentities:error_function_is_odd}
\end{equation}
as is easily determined by inspection.
Further, the error function \emph{commutes} with complex conjugation
\begin{equation}
  \erf(z^*) = [\erf(z)]^*
  \label{eq:PCIResponseIdentities:error_function_commutativity}
\end{equation}
where $z^*$ is the complex conjugate of $z$.


\section{Finite-domain inverse Fourier transform of a shifted Gaussian}
Each of the integrals in
(\ref{eq:PCIResponseIdentities:phase_plate_action_on_mth_beam_integrals})
can be easily evaluated by completing the square
of the exponential arguments to yield
\begin{equation}
  \begin{aligned}
    \int_{k_1}^{k_2}
    dk'
    e^{i k' x}
    e^{-\left[ \frac{w_0}{2} \left( k' - m k \right) \right]^2}
    =
    \frac{\sqrt{\pi}}{w_0}
    &e^{-(x / w_0)^2}
    e^{i m k x}
    \\
    &\times \delta(k_1, k_2, m)
  \end{aligned}
  \label{eq:PCIResponseIdentities:inverse_fourier_transform_shifted_Gaussian}
\end{equation}
where
\begin{equation}
  \delta(k_1, k_2, m)
  \equiv
  \erf[u(k_2, m)] - \erf[u(k_1, m)]
  \label{eq:PCIResponseIdentities:difference_of_error_functions}
\end{equation}
and
\begin{equation}
  u(k_j, m) \equiv \frac{w_0}{2}(k_j - m k) + i \frac{x}{w_0}
  \label{eq:PCIResponseIdentities:u}
\end{equation}


\section{Some useful symmetries and degeneracies}


\subsection{The error function}
By applying (\ref{eq:PCIResponseIdentities:error_function_is_odd}) and
(\ref{eq:PCIResponseIdentities:error_function_commutativity}),
it naturally follows that
\begin{equation}
  erf[u(-k_j, -m)] = - \, \{ erf[u(k_j, m)] \}^*
  \label{eq:PCIResponseIdentities:error_function_symmetry}
\end{equation}


\subsection{The difference between error functions}
By applying the error function's symmetry in
(\ref{eq:PCIResponseIdentities:error_function_symmetry}),
it naturally follows that
the difference between two error functions satisfies
\begin{equation}
  \delta(-k_1, -k_2, -m)
  =
  [\delta(k_2, k_1, m)]^*
  \label{eq:PCIResponseIdentities:difference_of_error_functions_symmetry}
\end{equation}


\subsection{The phase-plate operator $\mathcal{P}$}
Eq.~(\ref{eq:PCIResponseIdentities:inverse_fourier_transform_shifted_Gaussian})
allows (\ref{eq:PCIResponseIdentities:phase_plate_action_on_mth_beam_integrals})
to be rewritten as
\begin{equation}
  \begin{aligned}
    \mathcal{P}(x, k, m)
    &=
    C_m
    [%
      \delta(-k_D, -k_g, m)
      +
      \delta(k_g, k_D, m)
    ]
    \\
    &\quad+
    i \sqrt{\eta} C_m \delta(-k_g, k_g, m)
  \end{aligned}
  \label{eq:PCIResponseIdentities:phase_plate_action_on_mth_beam_simplified}
\end{equation}
where
\begin{equation}
  C_m
  =
  \frac{1}{2} \left[ e^{-(x / w_0)^2} e^{i m k x} \right]
\end{equation}
Note that $C_m$ is \emph{Hermitian};
that is, $C_{-m} = C_m^*$.
In fact,
(\ref{eq:PCIResponseIdentities:phase_plate_action_on_mth_beam_simplified})
can be decomposed into Hermitian $\mathcal{P}_H$ and
anti-Hermitian $\mathcal{P}_A$ components
\begin{equation}
  \mathcal{P}(x, k, m)
  =
  \mathcal{P}_H(x, k, m)
  +
  \mathcal{P}_A(x, k, m)
  \label{eq:PCIResponseIdentities:phase_plate_action_on_mth_beam_Hermitian_decomposed}
\end{equation}
where
\begin{align}
  \mathcal{P}_H(x, k, m)
  &\equiv
  C_m
  [%
    \delta(-k_D, -k_g, m)
    +
    \delta(k_g, k_D, m)
  ]
  \label{eq:PCIResponseIdentities:phase_plate_action_on_mth_beam_Hermitian}
  \\
  \mathcal{P}_A(x, k, m)
  &\equiv
  i \sqrt{\eta} C_m \delta(-k_g, k_g, m)
  \label{eq:PCIResponseIdentities:phase_plate_action_on_mth_beam_antiHermitian}
\end{align}
Application of
(\ref{eq:PCIResponseIdentities:difference_of_error_functions_symmetry})
readily shows that
\begin{align}
  \mathcal{P}_H(x, k, -m) &= [\mathcal{P}_H(x, k, m)]^*
  \\
  \mathcal{P}_A(x, k, -m) &= -[\mathcal{P}_A(x, k, m)]^*
\end{align}
Thus,
\begin{equation}
  \mathcal{P}(x, k, -m)
  =
  [\mathcal{P}_H(x, k, m)]^*
  -
  [\mathcal{P}_A(x, k, m)]^*
  \label{eq:PCIResponseIdentities:phase_plate_action_on_mth_beam_symmetry}
\end{equation}

Note that the above symmetry also implies a degeneracy when $m = 0$.
Specifically,
(\ref{eq:PCIResponseIdentities:phase_plate_action_on_mth_beam_Hermitian_decomposed})
states that
\begin{equation}
  \mathcal{P}(x, k, +0)
  =
  \mathcal{P}_H(x, k, 0)
  +
  \mathcal{P}_A(x, k, 0)
  \label{eq:PCIResponseIdentities:phase_plate_action_on_plus0_beam}
\end{equation}
while (\ref{eq:PCIResponseIdentities:phase_plate_action_on_mth_beam_symmetry})
shows that
\begin{equation}
  \mathcal{P}(x, k, -0)
  =
  [\mathcal{P}_H(x, k, 0)]^*
  -
  [\mathcal{P}_A(x, k, 0)]^*
  \label{eq:PCIResponseIdentities:phase_plate_action_on_minus0_beam}
\end{equation}
Of course, $\mathcal{P}(x, k, +0) = \mathcal{P}(x, k, -0)$; that is
\begin{equation}
  \mathcal{P}_H(x, k, 0)
  +
  \mathcal{P}_A(x, k, 0)
  =
  [\mathcal{P}_H(x, k, 0)]^*
  -
  [\mathcal{P}_A(x, k, 0)]^*
\end{equation}
Further, as $\mathcal{P}_H$ is dependent on $k_D$ but $\mathcal{P}_A$ is not,
$\mathcal{P}_H$ and $\mathcal{P}_A$ must be linearly independent;
that is
\begin{align}
  \mathcal{P}_H(x, k, 0) &= [\mathcal{P}_H(x, k, 0)]^*
  \\
  \mathcal{P}_A(x, k, 0) &= -[\mathcal{P}_A(x, k, 0)]^*
\end{align}
This proves that $\mathcal{P}_H(x, k, 0)$ is purely \emph{real} and
that $\mathcal{P}_A(x, k, 0)$ is purely \emph{imaginary}.

Another useful degeneracy occurs
when the fluctuation wavenumber vanishes (i.e.\ $k = 0$).
Note that $\mathcal{P}(x, k, m) = \mathcal{P}(x, m \cdot k)$;
that is, the $m$ and $k$ dependence of $\mathcal{P}$
only appears as the \emph{product} $(m \cdot k)$.
Just as $m = 0$ at finite $k$ yields $(m \cdot k) = 0$,
$k = 0$ at finite $m$ also gives $(m \cdot k) = 0$;
thus,
\begin{equation}
  \mathcal{P}(x, k=0, m) = \mathcal{P}(x, k, m=0)
  \label{eq:PCIResponseIdentities:phase_plate_action_on_wavenumber_0}
\end{equation}
Eq.~(\ref{eq:PCIResponseIdentities:phase_plate_action_on_wavenumber_0})
will be helpful in showing that the PCI response vanishes at $k = 0$.
