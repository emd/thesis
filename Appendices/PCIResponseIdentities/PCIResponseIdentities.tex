\chapter{Some identities for the PCI wavenumber response}
\label{app:PCIResponseIdentities}


\section{The phase-plate operator $\mathcal{P}$}
The effect of the phase plate on the $m$\ts{th} diffracted beam is given by
$\mathcal{P}(x, k, m)$ as defined in
(\ref{eq:InterferometricMethods:phase_plate_action_on_mth_beam});
$\mathcal{P}(x, k, m)$ is repeated here for completeness
\begin{equation}
  \mathcal{P}(x, k, m)
  \equiv
  \mathcal{F}_x^{-1}
  \left\{
    T_{\text{pp}}(k_x)
    \cdot
    e^{-\left[ \frac{w_0}{2} \left( k_x - m k \right) \right]^2}
  \right\}
  \label{eq:PCIResponseIdentities:phase_plate_action_on_mth_beam}
\end{equation}
where
\begin{equation}
  \begin{aligned}
    T_{\text{pp}}(k_x)
    &=
    i \sqrt{\eta} \, H(k_g - |k_x|)
    \\
    &\quad +
    H(|k_x| - k_g)
    H(k_D - |k_x|)
  \end{aligned}
  \label{eq:PCIResponseIdentities:phase_plate_transfer_function}
\end{equation}
is the phase-plate transfer function,
$H(x)$ is the Heaviside step function defined as
\begin{equation}
  H(x)
  =
  \begin{cases}
    0, \quad &x < 0 \\
    1, \quad &x \geq 0
  \end{cases}
  \label{eq:PCIResponseIdentities:Heaviside_step_function}
\end{equation}
and $k_g$ and $k_D$ are the low-$k$ and high-$k$ cutoffs of the phase plate
as defined in
(\ref{eq:InterferometricMethods:pci_kmin_engineering}) and
(\ref{eq:InterferometricMethods:pci_kmax_engineering}), respectively.
Thus,
\begin{align}
  \begin{aligned}
    \mathcal{P}(x, k, m)
    &=
    \frac{1}{2 \pi}
    \int_{-k_D}^{-k_g}
    dk_x
    e^{i k_x x}
    e^{-\left[ \frac{w_0}{2} \left( k_x - m k \right) \right]^2}
    \\
    &\quad +
    \frac{i \sqrt{\eta}}{2 \pi}
    \int_{-k_g}^{k_g}
    dk_x
    e^{i k_x x}
    e^{-\left[ \frac{w_0}{2} \left( k_x - m k \right) \right]^2}
    \\
    &\quad +
    \frac{1}{2 \pi}
    \int_{k_g}^{k_D}
    dk_x
    e^{i k_x x}
    e^{-\left[ \frac{w_0}{2} \left( k_x - m k \right) \right]^2}
  \end{aligned}
  \label{eq:PCIResponseIdentities:phase_plate_action_on_mth_beam_integrals}
\end{align}
Note that each of the above integrals
corresponds to a finite-domain inverse Fourier transform
of a shifted Gaussian.


\section{The error function}
The error function is defined as
\begin{equation}
  \erf(z)
  =
  \frac{2}{\sqrt{\pi}}
  \int_0^z e^{-t^2} dt
  \label{eq:PCIResponseIdentities:error_function}
\end{equation}
for complex argument $z$.
The error function is \emph{odd}
\begin{equation}
  \erf(-z) = - \erf(z)
  \label{eq:PCIResponseIdentities:error_function_is_odd}
\end{equation}
as is easily determined by inspection.
Further, the error function \emph{commutes} with complex conjugation
\begin{equation}
  \erf(z^*) = [\erf(z)]^*
  \label{eq:PCIResponseIdentities:error_function_commutativity}
\end{equation}
where $z^*$ is the complex conjugate of $z$.


\section{Finite-domain inverse Fourier transform of a shifted Gaussian}
Each of the integrals in
(\ref{eq:PCIResponseIdentities:phase_plate_action_on_mth_beam_integrals})
can be easily evaluated by completing the square
of the exponential arguments to yield
\begin{equation}
  \begin{aligned}
    \int_{k_1}^{k_2}
    dk'
    e^{i k' x}
    e^{-\left[ \frac{w_0}{2} \left( k' - m k \right) \right]^2}
    =
    \frac{\sqrt{\pi}}{w_0}
    &e^{-(x / w_0)^2}
    e^{i m k x}
    \\
    &\times \delta(k_1, k_2, m)
  \end{aligned}
  \label{eq:PCIResponseIdentities:inverse_fourier_transform_shifted_Gaussian}
\end{equation}
where
\begin{equation}
  \delta(k_1, k_2, m)
  \equiv
  \erf[u(k_2, m)] - \erf[u(k_1, m)]
  \label{eq:PCIResponseIdentities:difference_of_error_functions}
\end{equation}
and
\begin{equation}
  u(k_j, m) \equiv \frac{w_0}{2}(k_j - m k) + i \frac{x}{w_0}
  \label{eq:PCIResponseIdentities:u}
\end{equation}


\section{Some useful symmetries}


\subsection{The error function}
By applying (\ref{eq:PCIResponseIdentities:error_function_is_odd}) and
(\ref{eq:PCIResponseIdentities:error_function_commutativity}),
it naturally follows that
\begin{equation}
  erf[u(-k_j, -m)] = - \, \{ erf[u(k_j, m)] \}^*
  \label{eq:PCIResponseIdentities:error_function_symmetry}
\end{equation}


\subsection{The difference between error functions}
By applying the error function's symmetry in
(\ref{eq:PCIResponseIdentities:error_function_symmetry}),
it naturally follows that
the difference between two error functions satisfies
\begin{equation}
  \delta(-k_1, -k_2, -m)
  =
  [\delta(k_2, k_1, m)]^*
  \label{eq:PCIResponseIdentities:difference_of_error_functions_symmetry}
\end{equation}


\subsection{The phase-plate operator $\mathcal{P}$}
Eq.~(\ref{eq:PCIResponseIdentities:inverse_fourier_transform_shifted_Gaussian})
allows (\ref{eq:PCIResponseIdentities:phase_plate_action_on_mth_beam_integrals})
to be rewritten as
\begin{equation}
  \begin{aligned}
    \mathcal{P}(x, k, m)
    &=
    C_m
    [%
      \delta(-k_D, -k_g, m)
      +
      \delta(k_g, k_D, m)
    ]
    \\
    &\quad+
    i \sqrt{\eta} C_m \delta(-k_g, k_g, m)
  \end{aligned}
  \label{eq:PCIResponseIdentities:phase_plate_action_on_mth_beam_simplified}
\end{equation}
where
\begin{equation}
  C_m
  =
  \frac{e^{-(x / w_0)^2} e^{i m k x}}{2 \sqrt{\pi} w_0}
\end{equation}
Note that $C_m$ is \emph{Hermitian};
that is, $C_{-m} = C_m^*$.
In fact,
(\ref{eq:PCIResponseIdentities:phase_plate_action_on_mth_beam_simplified})
can be decomposed into Hermitian $\mathcal{P}_H$ and
anti-Hermitian $\mathcal{P}_A$ components
\begin{equation}
  \mathcal{P}(x, k, m)
  =
  \mathcal{P}_H(x, k, m)
  +
  \mathcal{P}_A(x, k, m)
  \label{eq:PCIResponseIdentities:phase_plate_action_on_mth_beam_Hermitian_decomposed}
\end{equation}
where
\begin{align}
  \mathcal{P}_H(x, k, m)
  &\equiv
  C_m
  [%
    \delta(-k_D, -k_g, m)
    +
    \delta(k_g, k_D, m)
  ]
  \label{eq:PCIResponseIdentities:phase_plate_action_on_mth_beam_Hermitian}
  \\
  \mathcal{P}_A(x, k, m)
  &\equiv
  i \sqrt{\eta} C_m \delta(-k_g, k_g, m)
  \label{eq:PCIResponseIdentities:phase_plate_action_on_mth_beam_antiHermitian}
\end{align}
Application of
(\ref{eq:PCIResponseIdentities:difference_of_error_functions_symmetry})
readily shows that
\begin{align}
  \mathcal{P}_H(x, k, -m) &= [\mathcal{P}_H(x, k, m)]^*
  \\
  \mathcal{P}_A(x, k, -m) &= -[\mathcal{P}_A(x, k, m)]^*
\end{align}
Thus,
\begin{equation}
  \mathcal{P}(x, k, -m)
  =
  [\mathcal{P}_H(x, k, m)]^*
  -
  [\mathcal{P}_A(x, k, m)]^*
  \label{eq:PCIResponseIdentities:phase_plate_action_on_mth_beam_symmetry}
\end{equation}

Note that the above symmetry also implies a degeneracy when $m = 0$.
Specifically,
(\ref{eq:PCIResponseIdentities:phase_plate_action_on_mth_beam_Hermitian_decomposed})
states that
\begin{equation}
  \mathcal{P}(x, k, +0)
  =
  \mathcal{P}_H(x, k, 0)
  +
  \mathcal{P}_A(x, k, 0)
  \label{eq:PCIResponseIdentities:phase_plate_action_on_plus0_beam}
\end{equation}
while (\ref{eq:PCIResponseIdentities:phase_plate_action_on_mth_beam_symmetry})
shows that
\begin{equation}
  \mathcal{P}(x, k, -0)
  =
  [\mathcal{P}_H(x, k, 0)]^*
  -
  [\mathcal{P}_A(x, k, 0)]^*
  \label{eq:PCIResponseIdentities:phase_plate_action_on_minus0_beam}
\end{equation}
Of course, $\mathcal{P}(x, k, +0) = \mathcal{P}(x, k, -0)$; that is
\begin{equation}
  \mathcal{P}_H(x, k, 0)
  +
  \mathcal{P}_A(x, k, 0)
  =
  [\mathcal{P}_H(x, k, 0)]^*
  -
  [\mathcal{P}_A(x, k, 0)]^*
\end{equation}
Further, as $\mathcal{P}_H$ is dependent on $k_D$ but $\mathcal{P}_A$ is not,
$\mathcal{P}_H$ and $\mathcal{P}_A$ must be linearly independent;
that is
\begin{align}
  \mathcal{P}_H(x, k, 0) &= [\mathcal{P}_H(x, k, 0)]^*
  \\
  \mathcal{P}_A(x, k, 0) &= -[\mathcal{P}_A(x, k, 0)]^*
\end{align}
This proves that $\mathcal{P}_H(x, k, 0)$ is purely \emph{real} and
that $\mathcal{P}_A(x, k, 0)$ is purely \emph{imaginary}.
