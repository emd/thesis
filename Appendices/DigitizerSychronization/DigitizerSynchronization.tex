\newcommand{\nom}{\text{nom}}


\chapter{Synchronization of digital records}
\label{app:DigitizerSynchronization}
\ldots
Efficient conversion of an analog signal to a digital record requires
quantization of the signal magnitude and
temporal sampling of these quantized magnitudes~\cite{bennett_bstj48}.


\section{Timebase of single digital record}
\label{app:DigitizerSynchronization:timebase_single_record}
Typically, temporal sampling of signal $x_j(t)$ occurs
at a fixed sampling rate $F_j$ such that
successive points in the digital record
are separated in time by $1 / F_j$.
Digitization begins at the ``trigger time'' $t_j[0]$ such that
the time corresponding to the $m\ts{th}$ digitized point is
\begin{equation}
  t_j[m] = t_j[0] + \frac{m}{F_j}.
  \label{eq:DigitizerSychronization:timebase_generic}
\end{equation}
Ideally, the \emph{realized} sampling rate $F_j$ and trigger time $t_j[0]$
are equal to their \emph{nominal} values
$F_j^{\nom}$ and $t_j^{\nom}[0]$, respectively.
However, short-term jitter, long-term drifts, and constant offsets
often plague real-world digitization such that
$F_j \neq F_j^{\nom}$, $t_j[0] \neq t_j^{\nom}[0]$, and
\begin{equation}
  t_j[m] \neq t_j^{\nom}[0] + \frac{m}{F_j^{\nom}};
\end{equation}
that is, the actual time base of the digital record
does \emph{not} equal the nominal timebase.
In a properly operating digitizer,
these discrepancies are typically small, and
an autospectral-density estimate (for example)
of $x_j(t)$ from its digital record
will be negligibly compromised.
When estimating the \emph{phasing}
between $x_j(t)$ and $x_{k}(t)$ for $j \neq k$, however,
identifying and correcting such timebase discrepancies
becomes paramount in importance.


\section{Which digital records can be synchronized?}
\label{app:DigitizerSynchronization:digitization_schemes}
The digitization scheme determines
whether or not digital records
$\{x_j[n]\}$ and $\{x_k[n]\}$
can be synchronized.
The cleanest, simplest, and most problem-free scheme
is to digitize $x_j(t)$ and $x_k(t)$ on the \emph{same} system
such that the actual sample rates and trigger times
of both digital records are identical
(i.e.\ $F_j = F_k$ and $t_j[0] = t_k[0]$, respectively).
However, such a scheme is not always feasible.
Further, note that multiple digitizer boards
operating in a master-slave configuration
can still suffer from trigger-time offsets,
despite nominally being part of the same digitization system.
The next-best scheme is to use phase-locked digitizers
such that $F_j / F_k = F_j^{\nom} / F_k^{\nom}$,
regardless of any short-term jitter or long-term drift
in the digitizer clocks.
In general, phase-locked digitizers
still suffer from trigger offsets, which result
from the digitizers having different actual trigger times
(i.e.\ $t_j[0] \neq t_k[0]$) and
from each digitizer triggering at
an actual time that differs from its nominal trigger time
(e.g.\ $t_j[0] \neq t_j^{\nom}[0]$).
As shown in
Section~\ref{app:DigitizerSynchronization:phase_locked_synchronization},
it \emph{is} possible to synchronize records
from phase-locked digitizers.
Finally, the least-desirable scheme
is to use free-running digitizers
such that $F_j / F_k \neq F_j^{\nom} / F_k^{\nom}$;
it may be impossible to synchronize records
from free-running digitizers.


\section{Synchronization of phase-locked digital records}
\label{app:DigitizerSynchronization:phase_locked_synchronization}
\begin{itemize}
  \item Trigger offset
  \item Estimating trigger offset from constant-frequency mode
  \item Estimating trigger offset from mode with linearly ramped frequency
\end{itemize}


\bibliographystyle{plainurl}
\bibliography{references}
