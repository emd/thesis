\newcommand{\nom}{\text{nom}}


\chapter{Synchronization of digital records}
\label{app:DigitizerSynchronization}
\ldots
Efficient conversion of an analog signal to a digital record requires
quantization of the signal magnitude and
temporal sampling of these quantized magnitudes~\cite{bennett_bstj48}.


\section{Timebase of single digital record}
Typically, temporal sampling of signal $x_j(t)$ occurs
at a fixed sampling rate $F_j$ such that
successive points in the digital record
are separated in time by $1 / F_j$.
Digitization begins at the ``trigger time'' $t_j[0]$ such that
the time corresponding to the $m\ts{th}$ digitized point is
\begin{equation}
  t_j[m] = t_j[0] + \frac{m}{F_j}.
  \label{eq:DigitizerSychronization:timebase_generic}
\end{equation}
Ideally, the \emph{realized} sampling rate $F_j$ and trigger time $t_j[0]$
are equal to their \emph{nominal} values
$F_j^{\nom}$ and $t_j^{\nom}[0]$, respectively.
However, short-term jitter, long-term drifts, and constant offsets
often plague real-world digitization such that
$F_j \neq F_j^{\nom}$, $t_j[0] \neq t_j^{\nom}[0]$, and
\begin{equation}
  t_j[m] \neq t_j^{\nom}[0] + \frac{m}{F_j^{\nom}};
\end{equation}
that is, the actual time base of the digital record
does \emph{not} equal the nominal timebase.
In a properly operating digitizer,
these discrepancies are typically small, and
an autospectral-density estimate (for example)
of $x_j(t)$ from its digital record
will be negligibly compromised.
When estimating the \emph{phasing}
between $x_j(t)$ and $x_{j + k}(t)$, however,
identifying and correcting such timebase discrepancies
becomes paramount in importance.


\section{Which digital records be synchronized?}


\section{Synchronization of phase-locked digital records}
\begin{itemize}
  \item Trigger offset
  \item Estimating trigger offset from constant-frequency mode
  \item Estimating trigger offset from mode with linearly ramped frequency
\end{itemize}


\bibliographystyle{plainurl}
\bibliography{references}
