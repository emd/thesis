\chapter{External Clock}
\label{app:ExternalClock}

\begin{itemize}
  \item Cite D-tAcq documentation
\end{itemize}

% The frequency division is performed in `Dt216Init.fun` via the commands
% 
%     set.ext_clk DIx falling
%     setExternalClock DIx [div DOy]
% 
% The first command maps the external clock to digital input line
% `DIx` (x = {0, 1, 2, ..., 5)} and tells the digitizer what clock
% characteristic to sample on (here, the falling edge (as opposed to
% the rising edge)). The second command tells the digitizer to actually
% use the external clocking (as opposed to internal) by accepting a clock
% on digital input line `DIx` (x = {0, 1, 2, ..., 5}), optionally deriving
% a "local" clock by dividing by integer factor `div`, and also optionally
% outputting the derived local clock to digital output line
% `DOy` (y = {0, 1, 2, ...5}, with the constraint that x != y).
% 
% Note that the *minimum* value of `div` is 2. Note further that the commands
% *MUST* be issued in the above order to get correct clocking between the master
% and slave boards (described below); this is not well-documented in the
% user manuals, but it is absolutely essential.
% 
% I've hard-coded `div` = 4 to obtain a 4 MS/s sampling rate when using a
% 16 MHz external clock.
% 
% Master and slave board:
% -----------------------
% Previously, board 7 generated a local internal clock and distributed this
% to board 8 via the PXI backplane; for this reason, board 7's clock source
% (node name: `CLOCK_SRC`) was referred to as "MASTER" in the MDSplus tree.
% The trigger was accepted on board 8 and similarly distributed to board 7.
% 
% Attempting to "streamline" the logic, I've now designated board 8 as "MASTER".
% This means that board 8 accepts both the external clock and the trigger and
% distributes both to board 7 (the "slave"). This required making changes to
% the MDSplus tree.
% 
% Signal routing:
% ---------------
% Signal routing was altered in `Dt216Init.fun` and the MDSplus tree to realize
% the external clock and master-and-slave configuration discussed above. There
% are also some very slight tweaks to how the routing is done in
% `dt216__init.fun` and `dt216__store.fun` that are well-explained in the
% commentary surrounding the adjacent changes.
% 
% Note that the signal routing and the fact that the minimum value of `div`
% in the setExternalClock command (above) means that the maximum sample rate
% we can achieve with a 16 MHz external clock is 8 MS/s. (Sampling at the full
% 16 MS/s requires non-trivial changes to the routing).
% 
% Tests:
% ------
% As mentioned above, Mike's clock system has two spare outputs, each
% programmed to output a 16 MHz clock signal. The first output is used
% as our external clock input. The second output was connected to our
% divide-by-4 flip-flop circuit and digitized on ch. 8 and 16. Sampling
% at 16 MS/s (which required manually re-routing the clock signals etc),
% we expect a strong peak exactly at 4 MHz (corresponding to the fundamental
% frequency of the 4 MHz square wave). This is exactly what we see,
% as shown in the attached figure (`4MHz_signal_at_16MSPS.pdf`).
% 
% Also, when sampling at <= 8 MS/s (i.e. the normal signal routing I've
% implemented above), the phase difference between the signals is constant
% (i.e. no relative drift between boards, as is desired. Note that sampling
% at 8 MS/s allows us to just barely resolve the phase of the 4 MHz square wave
% at the digitizer Nyquist frequency). For some reason, however, when
% digitizing at 16 MS/s (with the altered routing), the phase of the signals
% digitized between the two boards is *not* constant, presumably resulting
% from some subtlety of the digitizer operation that I'm not quite grasping;
% however, by restricting our sample rates to <= 8 MS/s and using the usual
% routing, we shouldn't run into any problems :)

